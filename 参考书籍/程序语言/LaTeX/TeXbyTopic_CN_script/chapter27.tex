% -*- coding: utf-8 -*-
\documentclass{book}

% -*- coding: utf-8 -*-

\usepackage[b5paper,text={5in,8in},centering]{geometry}
\usepackage{amsmath}
\usepackage[heading = false, scheme = plain]{ctex}
% \usepackage[CJKchecksingle]{xeCJK}
\setmainfont[Mapping=tex-text]{TeX Gyre Schola}
%\setsansfont{URW Gothic L Book}
%\setmonofont{Nimbus Mono L}
% \setCJKmainfont[BoldFont=FandolHei,ItalicFont=FandolKai]{FandolSong}
% \setCJKsansfont{FandolHei}
% \setCJKmonofont{FandolFang}
\xeCJKsetup{PunctStyle = kaiming}

\linespread{1.25}
\setlength{\parindent}{2em}
\setlength{\parskip}{0.5ex}
\usepackage{indentfirst}

\usepackage{xcolor}
\definecolor{myblue}{rgb}{0,0.2,0.6}

\usepackage{titlesec}
\titleformat{\chapter}
    {\normalfont\Huge\sffamily\color{myblue}}
    {第\thechapter 章}
    {1em}
    {}
%\titlespacing{\chapter}{0pt}{50pt}{40pt}
\titleformat{\section}
    {\normalfont\Large\sffamily\color{myblue}}
    {\thesection}
    {1em}
    {}
%\titlespacing{\section}{0pt}{3.5ex plus 1ex minus .2ex}{2.3ex plus .2ex}
\titleformat{\subsection}
    {\normalfont\large\sffamily\color{myblue}}
    {\thesubsection}
    {1em}
    {}
%\titlespacing{\subsection}{0pt}{3.25ex plus 1ex minus .2ex}{1.5ex plus .2ex}
%
\newpagestyle{special}[\small\sffamily]{
  \headrule
  \sethead[\usepage][][\chaptertitle]
  {\chaptertitle}{}{\usepage}}
\newpagestyle{main}[\small\sffamily]{
  \headrule
  \sethead[\usepage][][第\thechapter 章\quad\chaptertitle]
  {\thesection\quad\sectiontitle}{}{\usepage}}

\usepackage{titletoc}
%\setcounter{tocdepth}{1}
%\titlecontents{标题层次}[左间距]{上间距和整体格式}{标题序号}{标题内容}{指引线和页码}[下间距]
\titlecontents{chapter}[1.5em]{\vspace{.5em}\bfseries\sffamily}{\color{myblue}\contentslabel{1.5em}}{}
    {\titlerule*[20pt]{$\cdot$}\contentspage}[]
\titlecontents{section}[4.5em]{\sffamily}{\color{myblue}\contentslabel{3em}}{}
    {\titlerule*[20pt]{$\cdot$}\contentspage}[]
%\titlecontents{subsection}[8.5em]{\sffamily}{\contentslabel{4em}}{}
%    {\titlerule*[20pt]{$\cdot$}\contentspage}

\usepackage{enumitem}
\setlist{topsep=2pt,itemsep=2pt,parsep=1pt,leftmargin=\parindent}

\usepackage{fancyvrb}
\DefineVerbatimEnvironment{verbatim}{Verbatim}
  {xleftmargin=2em,baselinestretch=1,formatcom=\color{teal}\upshape}

\usepackage{etoolbox}
\makeatletter
\preto{\FV@ListVSpace}{\topsep=2pt \partopsep=0pt }
\makeatother

\usepackage[colorlinks,plainpages,pagebackref]{hyperref}
\hypersetup{
   pdfstartview={FitH},
   citecolor=teal,
   linkcolor=myblue,
   urlcolor=black,
   bookmarksnumbered
}

\usepackage{comment,makeidx,multicol}

%\usepackage{german}
%% german
%\righthyphenmin=3
%\mdqoff
%\captionsenglish
\usepackage[english]{babel}
{\catcode`"=13 \gdef"#1{\ifx#1"\discretionary{}{}{}\fi\relax}}
\def\mdqon{\catcode`"=13\relax}
\def\mdqoff{\catcode`"=12\relax}
\makeindex
\hyphenation{ex-em-pli-fies}

\newdimen\tempdima \newdimen\tempdimb

% these are fine
\def\handbreak{\\ \message{^^JManual break!!!!^^J}}
\def\nl{\protect\\}\def\n#1{{\tt #1}}
\protected\def\cs#1{\texttt{\textbackslash#1}}
\pdfstringdefDisableCommands{\def\cs#1{\textbackslash#1}}
\let\csc\cs
\def\lb{{\tt\char`\{}}\def\rb{{\tt\char`\}}}
\def\gr#1{\texorpdfstring{$\langle$#1$\rangle$}{<#1>}} %\def\gr#1{$\langle$#1$\rangle$}
\def\marg#1{{\tt \{}#1{\tt \}}}
\def\oarg#1{{\tt [}#1{\tt }}
\def\key#1{{\tt#1}}
\def\alt{}\def\altt{}%this way in manstijl
\def\ldash{\unskip\ ——\nobreak\ \ignorespaces}
\def\rdash{\unskip\nobreak\ ——\ \ignorespaces}
% check these
\def\hex{{\tt"}}
\def\ascii{{\sc ascii}}
\def\ebcdic{{\sc ebcdic}}
\def\IniTeX{Ini\TeX}\def\LamsTeX{LAMS\TeX}\def\VirTeX{Vir\TeX}
\def\AmsTeX{Ams\TeX}
\def\TeXbook{the \TeX\ book}\def\web{{\sc web}}
% needs major thinking
\newenvironment{myquote}{\list{}{%
    \topsep=2pt \partopsep=0pt%
    \leftmargin=\parindent \rightmargin=\parindent
    }\item[]}{\endlist}
\newenvironment{disp}{\begin{myquote}}{\end{myquote}}
\newenvironment{Disp}{\begin{myquote}}{\end{myquote}}
\newenvironment{tdisp}{\begin{myquote}}{\end{myquote}}
\newenvironment{example}{\begin{myquote}\noindent\itshape 例子:}{\end{myquote}}
\newenvironment{inventory}{\begin{description}\raggedright}{\end{description}}
\newenvironment{glossinventory}{\begin{description}}{\end{description}}
\def\gram#1{\gr{#1}}%???
\def\meta{\gr}% alias
%
% index
%
\def\indexterm#1{\emph{#1}\index{#1}}
\def\indextermsub#1#2{\emph{#1 #2}\index{#1!#2}}
\def\indextermbus#1#2{\emph{#1 #2}\index{#2!#1}}
\def\cindextermsub#1#2{\emph{#1#2}\index{#1!#1#2}}
\def\cindextermbus#1#2{\emph{#1#2}\index{#2!#1#2}}
\def\term#1\par{\index{#1}}
\def\howto#1\par{}
\def\cstoidx#1\par{\index{#1@\cs{#1}@}}
\def\thecstoidx#1\par{\index{#1@\protect\csname #1\endcsname}}
\def\thecstoidxsub#1#2{\index{#1, #2@\protect\csname #1\endcsname, #2}\ignorespaces}
\def\csterm#1\par{\cstoidx #1\par\cs{#1}}
\def\csidx#1{\cstoidx #1\par\cs{#1}}

\def\tmc{\tracingmacros=2 \tracingcommands\tracingmacros}

%%%%%%%%%%%%%%%%%%%
\makeatletter
\def\snugbox{\hbox\bgroup\setbox\z@\vbox\bgroup
    \leftskip\z@
    \bgroup\aftergroup\make@snug
    \let\next=}
\def\make@snug{\par\sn@gify\egroup \box\z@\egroup}
\def\sn@gify
   {\skip\z@=\lastskip \unskip
    \advance\skip\z@\lastskip \unskip
    \unpenalty
    \setbox\z@\lastbox
    \ifvoid\z@ \nointerlineskip \else {\sn@gify} \fi
    \hbox{\unhbox\z@}\nointerlineskip
    \vskip\skip\z@
    }

\newdimen\fbh \fbh=60pt % dimension for easy scaling:
\newdimen\fbw \fbw=60pt % height and width of character box

\newdimen\dh \newdimen\dw % height and width of current character box
\newdimen\lh % height of previous character box
\newdimen\lw \lw=.4pt % line weight, instead of default .4pt

\def\hdotfill{\noindent
    \leaders\hbox{\vrule width 1pt height\lw
                  \kern4pt
                  \vrule width.5pt height\lw}\hfill\hbox{}
    \par}
\def\hlinefill{\noindent
    \leaders\hbox{\vrule width 5.5pt height\lw         }\hfill\hbox{}
    \par}
\def\stippel{$\qquad\qquad\qquad\qquad$}
\makeatother
%%%%%%%%%%%%%%%%%%%

%\def\SansSerif{\Typeface:macHelvetica }
%\def\SerifFont{\Typeface:macTimes }
%\def\SansSerif{\Typeface:bsGillSans }
%\def\SerifFont{\Typeface:bsBaskerville }
\let\SansSerif\relax \def\italic{\it}
\let\SerifFont\relax \def\MainFont{\rm}
\let\SansSerif\relax
\let\SerifFont\relax
\let\PopIndentLevel\relax \let\PushIndentLevel\relax
\let\ToVerso\relax \let\ToRecto\relax

%\def\stop@command@suffix{stop}
%\let\PopListLevel\PopIndentLevel
%\let\FlushRight\relax
%\let\flushright\FlushRight
%\let\SetListIndent\LevelIndent
%\def\awp{\ifhmode\vadjust{\penalty-10000 }\else
%    \penalty-10000 \fi}
\let\awp\relax
\let\PopIndentLevel\relax \let\PopListLevel\relax

\showboxdepth=-1

%\input figs
\def\endofchapter{\vfill\noindent}

\newcommand{\liamfnote}[1]{\footnote{译注(Liam0205):#1}}
\newcommand{\cstate}[1]{状态 \textit{#1}}

\setcounter{chapter}{26}

\begin{document}

%\chapter{Page Breaking}\label{page:break}
\chapter{分页}\label{page:break}

%\index{page!breaking|(}
\index{page!breaking|(}

%This chapter treats the `page builder': the part of \TeX\
%that decides where to break the main vertical list into pages.
%The page builder operates before the output routine,
%and it hands its result in \cs{box255} to the output routine.
本章讨论`页面构建器':\TeX\ 确定在何处将主竖直列分为多个页面的模块。
页面构建器在输出例程之前运行,它将所得结果放入 \cs{box255} 以送给输出例程。

%\label{cschap:vsplit2}\label{cschap:splittopskip}\label{cschap:pagegoal}\label{cschap:pagetotal}\label{cschap:pagedepth}\label{cschap:pagestretch}\label{cschap:pagefilstretch}\label{cschap:pagefillstretch}\label{cschap:pagefilllstretch}\label{cschap:pageshrink}\label{cschap:outputpenalty}\label{cschap:interlinepenalty}\label{cschap:clubpenalty}\label{cschap:widowpenalty}\label{cschap:displaywidowpenalty}\label{cschap:brokenpenalty}\label{cschap:penalty2}\label{cschap:lastpenalty}\label{cschap:unpenalty}
%\begin{inventory}
%\item [\cs{vsplit}]
%      Split of a top part of a box. This is comparable
%      with page breaking.
\label{cschap:vsplit2}\label{cschap:splittopskip}\label{cschap:pagegoal}\label{cschap:pagetotal}\label{cschap:pagedepth}\label{cschap:pagestretch}\label{cschap:pagefilstretch}\label{cschap:pagefillstretch}\label{cschap:pagefilllstretch}\label{cschap:pageshrink}\label{cschap:outputpenalty}\label{cschap:interlinepenalty}\label{cschap:clubpenalty}\label{cschap:widowpenalty}\label{cschap:displaywidowpenalty}\label{cschap:brokenpenalty}\label{cschap:penalty2}\label{cschap:lastpenalty}\label{cschap:unpenalty}
\begin{inventory}
\item [\cs{vsplit}]
      分割出一个盒子的顶端部分。它可与分页作对比。

%\item [\cs{splittopskip}] 
%      Minimum distance between the top of what remains after a
%      \cs{vsplit} operation, and the first item in that box.
%      Plain \TeX\ default:~\n{10pt}
\item [\cs{splittopskip}] 
      在 \cs{vsplit} 分割出的盒子中,第一个项目到盒子顶部的最小距离。
      Plain \TeX\ 默认为~\n{10pt}。

%\item [\cs{pagegoal}] 
%      Goal height of the page box. This starts at \cs{vsize},
%      and is diminished by heights of insertion items.
\item [\cs{pagegoal}] 
      页面盒子的目标高度。它刚开始等于 \cs{vsize},
      并且根据插入项的高度逐步减少。

%\item [\cs{pagetotal}] 
%      Accumulated natural height of the current page.
\item [\cs{pagetotal}] 
      当前页面积累的自然高度。

%\item [\cs{pagedepth}] 
%      Depth of the current page.
\item [\cs{pagedepth}] 
      当前页面的深度。

%\item [\cs{pagestretch}] 
%      Accumulated zeroth-order stretch of the current page.
\item [\cs{pagestretch}] 
      当前页面积累的零阶可伸长量。

%\item [\cs{pagefilstretch}] 
%      Accumulated first-order stretch of the current page.
\item [\cs{pagefilstretch}] 
      当前页面积累的一阶可伸长量。

%\item [\cs{pagefillstretch}] 
%      Accumulated second-order stretch of the current page.
\item [\cs{pagefillstretch}] 
      当前页面积累的二阶可伸长量。

%\item [\cs{pagefilllstretch}] 
%      Accumulated third-order stretch of the current page.
\item [\cs{pagefilllstretch}] 
      当前页面积累的三阶可伸长量。

%\item [\cs{pageshrink}] 
%      Accumulated shrink of the current page.
\item [\cs{pageshrink}] 
      当前页面积累的可收缩量。

%\item [\cs{outputpenalty}]   
%      Value of the penalty at the current page break,
%      or $10\,000$ if the break was not at a penalty.
\item [\cs{outputpenalty}]   
      在当前分页点的惩罚值,若不在惩罚处分页,它等于 $10\,000$。

%\item [\cs{interlinepenalty}] 
%      Penalty for breaking a page between lines of a paragraph. 
%      Plain \TeX\ default:~\n{0}
\item [\cs{interlinepenalty}] 
      在段落内部两行间分页的惩罚值。Plain \TeX\ 默认为~\n{0}。

%\item [\cs{clubpenalty}] 
%      Additional penalty for breaking a page after 
%      the first line of a paragraph. 
%      Plain \TeX\ default:~\n{150}
\item [\cs{clubpenalty}] 
      在段落首行之后分页的额外惩罚值。Plain \TeX\ 默认为~\n{150}。

%\item [\cs{widowpenalty}] 
%      Additional penalty for breaking a page before 
%      the last line of a paragraph. 
%      Plain \TeX\ default:~\n{150}
\item [\cs{widowpenalty}] 
      在段落尾行之前分页的额外惩罚值。Plain \TeX\ 默认为~\n{150}。

%\item [\cs{displaywidowpenalty}] 
%      Additional penalty for breaking a page before the last line 
%      above a display formula. 
%      Plain \TeX\ default:~\n{50}
\item [\cs{displaywidowpenalty}] 
      在陈列公式上面的尾行之前分页的额外惩罚值。Plain \TeX\ 默认为~\n{50}。

%\item [\cs{brokenpenalty}] 
%      Additional penalty for breaking a page after a hyphenated line. 
%      Plain \TeX\ default:~\n{100}
\item [\cs{brokenpenalty}] 
      在连字行之后分页的额外惩罚值。Plain \TeX\ 默认为~\n{100}。

%\item [\cs{penalty}]
%      Place a penalty on the current list.
%\item [\cs{lastpenalty}]
%      If the last item on the list was a penalty, the value of this.
%\item [\cs{unpenalty}]
%      Remove the last item of the current list if this
%      was a penalty.
\item [\cs{penalty}]
      在当前列表上添加一个惩罚项。
\item [\cs{lastpenalty}]
      如果当前列表的上一项为惩罚项,它表示该惩罚项的值。
\item [\cs{unpenalty}]
      如果当前列表的上一项为惩罚项,删除该项目。

%\end{inventory}
\end{inventory}

%%\point The current page and the recent contributions
%\section{The current page and the recent contributions}
%\point The current page and the recent contributions
\section{当前页面与备选内容}

%The main vertical list of \TeX\ is divided in two parts:
%the \indexterm{current page} and the list of \indexterm{recent contributions}.
%Any material that is added to the main vertical list is
%appended to the recent contributions; the act of moving
%the recent contributions to the current page is known
%as exercising the \indextermsub{page}{builder}.
\TeX\ 的主竖直列被分为两部分:\indexterm{当前页面}和\indexterm{备选内容}列。
任何添加到主竖直列的素材被追加到备选内容中;
将素材从备选内容移动到当前页面的操作称为执行\indextermsub{页面}{构建器}.

%Every time something is moved to the current page, \TeX\
%computes the cost of breaking the page at that point.
%If it decides that it is past the optimal point,
%the current page up to 
%\altt 
%the best break so far
%is put in \cs{box255} and the remainder of
%the current page is moved back on top of the recent contributions.
%If the page is broken at a penalty,
%\label{break:penalty}%
%that value is recorded in \cs{outputpenalty}, and 
%a penalty of size $10\,000$ is placed on top of the
%recent contributions; otherwise, \csidx{outputpenalty}
%is set to~$10\,000$.
每次有内容移动到当前页面时,\TeX\ 计算在该处分页的代价。
如果 \TeX\ 判断该处已经越过最佳分页点,
当前页面在最佳分页点之前的所有内容就被放入\cs{box255},
\altt 
而剩下的内容被放回备选内容顶部。如果页面在惩罚项处分页,
\label{break:penalty}%
该惩罚的值就被记录在 \cs{outputpenalty} 中,
而且值为 $10\,000$ 的惩罚项被放在备选内容的顶部;
否则,\csidx{outputpenalty} 就被设为~$10\,000$。

%If the current page is empty, discardable items that are moved
%from the recent contributions are discarded. This is the mechanism
%that lets glue disappear after a page break and at the top of
%the first page. When the first non-discardable item is moved
%to the current page, the \cs{topskip} glue is inserted;
%see the previous chapter.
如果当前页面是空的,从备选内容移动过来的可弃项目将被丢弃。
此机制会让分页点之后以及第一页顶部的粘连消失。
在第一个非可弃项目移动到当前页面时,\cs{topskip} 粘连将会被插入;
详见上一章。

%The workings of the page builder can be made visible by
%setting \cs{tracingpages} to some positive value
%(see Chapter~\ref{trace}).
要看到页面构建器的工作过程,可以将 \cs{tracingpages}
设定为某个正值(见第~\ref{trace}~章)。

%%\point Activating the page builder
%\section{Activating the page builder}
%\point Activating the page builder
\section{激活页面构建器}

%The page builder comes into play  in the
%following circumstances.
%\begin{itemize}\item Around paragraphs: after the \cs{everypar}
%    tokens have been inserted, and after the paragraph has been
%    added to the vertical list. See the end of this chapter for 
%    an example.
%\item Around display formulas: after the \cs{everydisplay}
%    tokens have been inserted, and after the display has been
%    added to the list.
%\item After \cs{par} commands, boxes, insertions, 
%    and explicit penalties in vertical mode. 
%\item After an output routine has ended. \end{itemize}
%In these places the page builder moves the recent
%contributions to the current page. Note that \TeX\ need not be 
%in vertical mode when the page builder is exercised.
%In horizontal mode, activating the page builder
%serves to move preceding vertical glue (for example, \cs{parskip},
%\cs{abovedisplayskip}) to the page.
页面构建器将在下列位置起作用:
\begin{itemize}
\item 在段落前后:在 \cs{everypar} 记号列被插入后,
  以及段落被加入到竖直列后。见本章结尾处的例子。
\item 在陈列公式前后:在 \cs{everydisplay} 记号列被插入后,
  以及陈列公式被加入到竖直列后。
\item 在竖直模式的 \cs{par} 命令、盒子、插入项和显式惩罚项之后。
\item 在输出例程结束后。
\end{itemize}
页面构建器在这些位置将备选内容移动到当前页面。
注意页面构建器运行时 \TeX\ 无需处于竖直模式中。
在水平模式,激活页面构建器是为了将前面的竖直粘连(比如 \cs{parskip}
和 \cs{abovedisplayskip})移动到当前页面。

%The \cs{end} command \ldash which is only allowed in
%external vertical mode \rdash  terminates a \TeX\ job, but only if the
%main vertical list is empty and \cs{deadcycles}${}=0$.
%If this is not the case the combination
%\label{end:play}%
%\begin{disp}\verb>\hbox{}\vfill\penalty>$-2^{30}$\end{disp}
%is appended, which forces the output routine to act.
\cs{end} 命令 \ldash 它只能用于外部竖直模式中 \rdash
在主竖直列为空且 \cs{deadcycles}${}=0$ 时将终止 \TeX\ 任务。
否则,下列这串记号
\label{end:play}%
\begin{disp}\verb>\hbox{}\vfill\penalty>$-2^{30}$\end{disp}
将被插入进来,这促使输出例程开始运行
【译注:然后 \cs{end} 命令将被重新读入】。

%%\point Page length bookkeeping
%\section{Page length bookkeeping}
%\point Page length bookkeeping
\section{页面长度的记录}

%\index{page!length|(}
\index{page!length|(}

%The height and depth of the page box that reaches the output
%routine are determined by \cs{vsize}, \cs{topskip},
%and~\cs{maxdepth} as described in the previous chapter.
%\TeX\ places the \cs{topskip} glue
%when the first box is placed on the current page; the
%\cs{vsize} and \cs{maxdepth} are read when the first
%box or insertion occurs on the page. Any subsequent changes to these
%parameters will not be noticeable until the next page or,
%more strictly, until after the output routine has been called.
如上一章所述,到达输出例程的页面盒子的高度和深度由 \cs{vsize}、
\cs{topskip} 和 \cs{maxdepth} 确定。在第一个盒子出现当前页面时,
\TeX\ 放入 \cs{topskip} 粘连;而 \cs{vsize} 和 \cs{maxdepth}
在第一个盒子或插入项出现在页面时读取。此后对这些参数的修改将不会被注意到,
直到下一页或者,更严格地说,直到输出例程调用完毕之后。

%After the first box, rule, or insertion on the current page
%the \cs{vsize} is recorded in \cs{pagegoal},
%and its value is not looked at  until \cs{output}
%has been active.
%Changing \cs{pagegoal} does have an effect on the current
%page.
%When the page is empty,
%the pagegoal is \cs{maxdimen}, and \cs{pagetotal} is zero.
在第一个盒子、标线或插入项出现在当前页面之后,
\cs{vsize} 被记录在 \cs{pagegoal} 中,
并且在 \cs{output} 调用完毕之前不再被查看。
改变 \cs{pagegoal} 对当前页面不会有任何影响。
当页面为空时,\cs{pagegoal} 等于 \cs{maxdimen},而 \cs{pagetotal} 等于零。

%Accumulated dimensions and stretch are available in
%the parameters \cs{pagetotal}, \cs{pagedepth}, 
%\cs{pagestretch}, \cs{pagefilstretch}, \cs{pagefillstretch}, 
%\cs{pageshrink},
%and \cs{pagefilllstretch}.
%\cstoidx pagetotal\par\cstoidx pagedepth\par
%\cstoidx pagestretch\par\cstoidx pagefilstretch\par
%\cstoidx pagefillstretch\par
%\cstoidx pageshrink\par\cstoidx pagefilllstretch\par
%They are set by the page builder. The stretch and
%shrink parameters are updated every time glue is added
%to the page. The depth parameter becomes zero
%if the last item was kern or glue.
积累的尺寸和可伸缩量记录在 \cs{pagetotal}、\cs{pagedepth}、
\cs{pagestretch}、\cs{pagefilstretch}、\cs{pagefillstretch}、
\cs{pagefilllstretch} 以及 \cs{pageshrink}这些参数中。
\cstoidx pagetotal\par\cstoidx pagedepth\par
\cstoidx pagestretch\par\cstoidx pagefilstretch\par
\cstoidx pagefillstretch\par
\cstoidx pageshrink\par\cstoidx pagefilllstretch\par
它们由页面构建器设定。在添加每个粘连到页面时,
可伸长量和可收缩量参数都会被更新。
如果最后一个项目是紧排或者粘连,\cs{pagedepth} 参数等于零。

%These parameters are \gr{special dimen}s; an assignment
%to any of them is an \gr{intimate assignment},
%and it is automatically global.
这些参数都是 \gr{special dimen};对它们中任何一个的赋值都是
\gr{intimate assignment},而且自动就是全局的。

%\index{page!length|)}
\index{page!length|)}

%\section{Breakpoints}
\section{分页点}

%%\spoint Possible breakpoints
%\subsection{Possible breakpoints}
%\spoint Possible breakpoints
\subsection{可能的分页点}

%Page breaks can occur at the same kind of locations where
%line breaks can occur. The \indexterm{breakpoints in vertical lists}
%are:
%\begin{itemize}\item at glue that is preceded by a non-discardable
%item;\item at a kern that is immediately followed by glue;
%\item at a penalty.\end{itemize}
%\TeX\ inserts interline glue and various sorts of
%interline penalties when the lines of a paragraph are
%added to the vertical list, so there will usually be 
%sufficient breakpoints on the page.
分页点可以出现在与断行点相同类型的位置。
\indexterm{竖直模式的断点}如下:
\begin{itemize}
\item 在粘连处,只要它前面是非可弃项目;
\item 在紧排处,只要它后面是一个粘连;
\item 在惩罚处。
\end{itemize}
在添加段落行到竖直列时,\TeX\ 会插入行间粘连和各种行间惩罚,
这使得页面上通常有足够的分页点。

%%\spoint Breakpoint penalties
%\subsection{Breakpoint penalties}
%\spoint Breakpoint penalties
\subsection{分页点的惩罚}

%If \TeX\ decides to break a page at a penalty item, this
%penalty will, most of the time, be one that
%has been inserted automatically
%between the lines of a paragraph.
如果 \TeX\ 决定在惩罚项处分页,这个惩罚项,在多数情况下,是之前自动插入到段落各行间的。

%If the last item on a list (not necessarily a vertical list)
%\alt
%is a penalty, the value of this is recorded
%in the parameter \csidx{lastpenalty}. If the item is other than
%a penalty, this parameter has the value zero.
%The last penalty of a list can be removed with the command
%\csidx{unpenalty}. See Section~\ref{varioset} for an example.
%\message{Spoint ref varioset}
如果列表(未必得是竖直列)的最后一项是个惩罚项,
\alt
它的值会在记录在 \csidx{lastpenalty} 参数中。
如果这个项目不是一个惩罚项,这个参数的值为零。
列表的最后一个惩罚项可以用命令 \csidx{unpenalty} 删除。
见第~\ref{varioset}~节的例子。
\message{Spoint ref varioset}

%Here is a list of the \indextermsub{penalties}{in vertical mode}:
%\begin{inventory}
%\item [\cs{interlinepenalty}] 
%      Penalty for breaking a page between lines of a paragraph. 
%      In plain \TeX\ this is zero, so no penalty is added in
%      between lines. \TeX\ can then find a valid breakpoint at the
%      \cs{baselineskip} glue.
下面列出了各种\emph{竖直模式的惩罚}\index{惩罚!竖直模式中}:
\begin{inventory}
\item [\cs{interlinepenalty}] 
      在段落内部两行间分页的惩罚值。 
      在 plain \TeX\ 中它默认为零,因此两行间不会添加惩罚项。
      这样 \TeX\ 可以在 \cs{baselineskip} 粘连处寻找有效分页点。

%\item [\cs{clubpenalty}] 
%      Extra penalty for breaking a page after the first line of a paragraph. 
%      In plain \TeX\ this is~\n{150}.
%      This amount, and the following penalties, are 
%      added to the \cs{interlinepenalty}, and
%      a penalty of the resulting size is inserted after the
%      \cs{hbox} containing the first line of a paragraph
%      instead of the \cs{interlinepenalty}.
\item [\cs{clubpenalty}] 
      在段落首行之后分页的额外惩罚值。Plain \TeX\ 默认为~\n{150}。
      这个数值将被加到 \cs{interlinepenalty} 上,
      依照所得结果插入一个惩罚项到包含段落第一行的 \cs{hbox} 之后,
      以替代 \cs{interlinepenalty}。后面几个惩罚的情形类似。

%\item [\cs{widowpenalty}] 
%      Extra penalty for breaking a page before the last line of a paragraph. 
%      In plain \TeX\ this is~\n{150}.
\item [\cs{widowpenalty}] 
      在段落尾行之前分页的额外惩罚值。 Plain \TeX\ 默认为~\n{150}。

%\item [\cs{displaywidowpenalty}] 
%      Extra penalty for breaking a page before the last line 
%      above a display formula. The default value in plain \TeX\
%      is~\n{50}.
\item [\cs{displaywidowpenalty}] 
      在陈列公式上面的尾行之前分页的额外惩罚值。在 plain \TeX\ 中默认为~\n{50}。

%\item [\cs{brokenpenalty}] 
%      Extra penalty for breaking a page after a hyphenated line. 
%      The default value in plain \TeX\ is~\n{100}.
%\end{inventory}
%If the resulting penalty is zero, it is not placed.
\item [\cs{brokenpenalty}] 
      在连字行之后分页的额外惩罚值。在 plain \TeX\ 中默认为~\n{100}。
\end{inventory}
如果惩罚值等于零,惩罚项将不会被插入。

%Penalties can also be inserted by the user. For instance,
%the plain format has macros to encourage (possibly, force)
%or prohibit page breaks\cstoidx penalty\par:
%\begin{verbatim}
%\def\break{\penalty-10000 }        % force break
%\def\nobreak{\penalty10000 }       % prohibit break
%\def\goodbreak{\par\penalty-500 }  % encourage page break
%\end{verbatim}
%Also, \verb>\vadjust{\penalty ... }> is a way of getting
%penalties in the vertical list. This can be used to
%discourage or encourage page breaking after a certain
%line of a paragraph.
用户也可以插入惩罚项。比如 plain 格式有用于强制、阻止或鼓励分页的%
\cstoidx 惩罚\par 宏:
\begin{verbatim}
\def\break{\penalty-10000 }        % force break
\def\nobreak{\penalty10000 }       % prohibit break
\def\goodbreak{\par\penalty-500 }  % encourage page break
\end{verbatim}
另外,\verb>\vadjust{\penalty ... }> 是另一种在竖直列中添加惩罚项的方法。
它可以用于阻碍或鼓励在段落某行之后分页。

%%\spoint Breakpoint computation
%\subsection{Breakpoint computation}
%\spoint Breakpoint computation
\subsection{分页点的计算}

%\index{breakpoints!computation of|(}
%\advance\rightskip by 5.5cm
\index{breakpoints!computation of|(}
\advance\rightskip by 5.5cm

%Whenever an item is moved to the current page, \TeX\
%\vadjust{\advance\hsize by -5.5cm
%  \hbox to \hsize{\hfil\rlap{\hskip.4cm\vtop to 0pt
%  {\kern-2\baselineskip
%    \SansSerif %\pointSize:8 \Style:roman
%   \parindent0pt \offinterlineskip
%   \def\tbox#1{\hbox{\quad\quad #1%
%               \vrule height 10pt depth3pt width0cm }}
%   \hbox
%   {\vrule width\lw \kern-\lw
%    \vbox{\hsize=5cm
%          \hrule height\lw \ \vskip0cm
%           \kern40pt
%           \tbox{underfull page}
%           \tbox{$b=10\,000$}
%           \kern40pt
%          \hrule height\lw
%           \kern8pt
%           \tbox{feasible breakpoints}
%           \tbox{$b<10\,000$}
%           \kern8pt
%          \hrule height\lw
%           \kern8pt
%           \tbox{overfull page}
%           \tbox{$b=\infty$}
%           \kern3pt
%           \tbox{.\vrule height3.5pt depth1pt width0cm}
%           \tbox{.\vrule height3.5pt depth1pt width0cm}
%           \tbox{.\vrule height3.5pt depth1pt width0cm}
%           \kern8pt
%           }%
%    \kern-\lw \vrule width\lw}%
%    \vss}}}}
%computes the penalty $p$ and the badness $b$ associated with
%breaking the page at that place. From the penalty and
%the badness the cost $c$ of breaking is computed.
每当有项目移动到当前页面,\TeX\
\vadjust{\advance\hsize by -5.5cm
  \hbox to \hsize{\hfil\rlap{\hskip.4cm\vtop to 0pt
  {\kern-2\baselineskip
    \SansSerif %\pointSize:8 \Style:roman
   \parindent0pt \offinterlineskip
   \def\tbox#1{\hbox{\quad\quad #1%
               \vrule height 10pt depth3pt width0cm }}
   \hbox
   {\vrule width\lw \kern-\lw
    \vbox{\hsize=5cm
          \hrule height\lw \ \vskip0cm
           \kern40pt
           \tbox{未满的页面}
           \tbox{$b=10\,000$}
           \kern40pt
          \hrule height\lw
           \kern8pt
           \tbox{可行的断点}
           \tbox{$b<10\,000$}
           \kern8pt
          \hrule height\lw
           \kern8pt
           \tbox{过满的页面}
           \tbox{$b=\infty$}
           \kern3pt
           \tbox{.\vrule height3.5pt depth1pt width0cm}
           \tbox{.\vrule height3.5pt depth1pt width0cm}
           \tbox{.\vrule height3.5pt depth1pt width0cm}
           \kern3pt
           }%
    \kern-\lw \vrule width\lw}%
    \vss}}}}
计算在该位置分页时的惩罚值 $p$ 和丑度 $b$。
从惩罚值和丑度再计算出分页的代价 $c$。

%The place of least cost is remembered, and when
%the cost is infinite, that is, the page is overfull, or
%when the penalty is $p\leq-10\,000$, the current page is broken
%at  the (last remembered) place of least cost. 
%The broken-off piece is then
%put in \cs{box255} and the output routine token list
%is inserted. Box 255 is always given a height of \cs{vsize},
%regardless of how much material it has.
代价最小的位置被记录下来,而当代价为无穷时,即页面过满或惩罚值为
$p\leq-10\,000$ 时,当前页面就在(上回记录的)代价最小位置分开。
分出来的部分就被放入 \cs{box255},然后输出例程记号列被插入。
第~255~号盒子的高度总是被设为 \cs{vsize},不管它包含多少素材。

%The badness calculation is based on the amount of stretching
%or shrinking that is necessary to fit the page in
%a box with height \cs{vsize} 
%and maximum depth \cs{maxdepth}. This calculation is
%the same as for line breaking (see Chapter~\ref{glue}).
%Badness is a value $0\leq b\leq 10\,000$, except when
%pages are overfull; then~$b=\infty$.
丑度计算基于要将页面放入高度为 \cs{vsize} 深度不超过 \cs{maxdepth}
的盒子所需的伸长或收缩量。计算方法和断行时一样(见第~\ref{glue}~章)。
丑度满足 $0\leq b\leq 10\,000$,除了在页面过满时~$b=\infty$。

%\advance\rightskip by -5.5cm
\advance\rightskip by -5.5cm

%Some penalties are implicitly inserted by \TeX, 
%for instance the \cs{interlinepenalty}
%which is put in between every pair of lines of a paragraph.
%Other penalties can
%be explicitly inserted by the user or a user macro. 
%A~penalty
%value $p\geq10\,000$ inhibits breaking; a penalty
%$p\leq-10\,000$ (in external vertical mode)
%\alt
%forces a page break, and immediately
%activates the output routine. 
有些惩罚是由 \TeX\ 隐式插入的,比如 \cs{interlinepenalty}
就是在段落任何两行间自动插入的。其他惩罚可以由用户或用户宏隐式插入。
惩罚值 $p\geq10\,000$ 将阻止分页;而惩罚值
$p\leq-10\,000$(在外部竖直模式中)%
\alt
将强制分页,并立即激活输出例程。

%Cost calculation proceeds as follows:
%\begin{enumerate} \item When a penalty is so low that it forces
%a page break and immediate invocation of the output routine, 
%but the page is not overfull, that is
%\begin{disp}$b<\infty\quad\hbox{and}\quad p\leq-10\,000$\end{disp}
%the cost is equal to the penalty:~$c=p$.
分页代价依照下面步骤计算:
\begin{enumerate}
\item 若惩罚值小到导致强制分页并立即激活输出例程,而且页面不是过满的,即
\begin{disp}$b<\infty\quad\hbox{且}\quad p\leq-10\,000$\end{disp}
则代价等于惩罚值:$c=p$。

%\item When penalties do not force anything, and the page is not
%overfull, that is
%\begin{disp}$b<\infty\quad\hbox{and}\quad |p|<10\,000$\end{disp}
%the cost is~$c=b+p$.
\item 若惩罚值不会导致强制或禁止分页,而且页面不是过满的,即
\begin{disp}$b<\infty\quad\hbox{且}\quad |p|<10\,000$\end{disp}
则代价等于~$c=b+p$。

%\item For pages that are very bad, that is
%\begin{disp}$b=10\,000\quad\hbox{and}\quad |p|<10\,000$\end{disp}
%the cost is~$c=10\,000$.
\item 若页面非常糟糕,即
\begin{disp}$b=10\,000\quad\hbox{且}\quad |p|<10\,000$\end{disp}
则代价等于~$c=10\,000$.

%\item An overfull page, that is
%\begin{disp}$b=\infty\quad\hbox{and}\quad p<10\,000$\end{disp}
%gives infinite cost:~$c=\infty$.
%In this case \TeX\ decides that the optimal break point
%must have occurred earlier, and it invokes the output routine.
%Values of \cs{insertpenalties} (see Chapter~\ref{insert})
%that exceed $10\,000$
%also give infinite cost.
%\end{enumerate}
\item 若页面过满,即
\begin{disp}$b=\infty\quad\hbox{且}\quad p<10\,000$\end{disp}
则代价为无穷大:$c=\infty$.
此时 \TeX\ 判定最佳分页点应该出现在前面,并且调用输出例程。
超过 $10\,000$ 的 \cs{insertpenalties} 值(见第~\ref{insert}~章)%
同样给出无穷大的代价。
\end{enumerate}

%The fact that a penalty $p\leq-10\,000$ activates
%the output routine is used extensively
%in the \LaTeX\ output routine: 
%the excess $\mathopen|p\mathclose|-10\,000$ is
%a code indicating the reason for calling the output routine;
%see also the second example in the next chapter.
惩罚值 $p\leq-10\,000$ 将激活输出例程这个事实在 \LaTeX\ 输出例程中多次用到:
差值码 $\mathopen|p\mathclose|-10\,000$ 用于表示调用输出例程的原因%
【译注:见 \verb|source2e| 第~317~页】;另外可以见下一章的第二个例子。

%\index{breakpoints!computation of|)}
\index{breakpoints!computation of|)}

%\section{\protect\cs{vsplit}}
%\label{vsplit}
\section{分割竖直列}
\label{vsplit}

%The page-breaking operation is available to the user
%through the \csidx{vsplit} operation.
用户可以通过 \csidx{vsplit} 命令实现分页操作。

%\begin{example}
%\begin{verbatim}
%\setbox1 = \vsplit2 to \dimen3
%\end{verbatim}
%assigns to box~1 the top part of size \cs{dimen3}
%of box~2. This material is actually removed from box~2.
%Compare this with splitting off a chunk of size \cs{vsize}
%from the current page.
%\end{example}
\begin{example}
\begin{verbatim}
\setbox1 = \vsplit2 to \dimen3
\end{verbatim}
将~2~号盒子顶部 \cs{dimen3} 大小的部分赋予~1~号盒子。
这部分素材实际上是从~2~号盒子移除出来的。
可以将它与从当前页面分割出 \cs{vsize} 大小的一块内容作对比。
\end{example}

%The extracted
%result of 
%\begin{disp}\cs{vsplit}\gr{8-bit number}\n{to}\gr{dimen}
%\end{disp} is a box with the following properties.
%\begin{itemize} \item Height equal to the specified \gr{dimen}; \TeX\ will
%      go through the original box register (which must contain
%      a vertical box) to find the best breakpoint. This may
%      result in an underfull box.
%\item Depth at most \csidx{splitmaxdepth}; this is analogous to
%            the \cs{maxdepth} for the page box, rather than the \cs{boxmaxdepth}
%      that holds for any box.
%\item A first and last mark in the \cs{splitfirstmark} and 
%      \cs{splitbotmark} registers.
%\end{itemize}
分割命令
\begin{disp}
\cs{vsplit}\gr{8-bit number}\n{to}\gr{dimen}
\end{disp}
提取出的结果是一个满足下列属性的盒子:
\begin{itemize}
\item 高度等于指定的 \gr{dimen};\TeX\ 将遍历原始的盒子寄存器%
      (它必须容纳一个竖直盒子)以找到最佳断点。最后可能得到一个未满盒子。
\item 深度不超过 \csidx{splitmaxdepth};与它类似的是页面盒子的 \cs{maxdepth},
      而不是对任何盒子都适用的 \cs{boxmaxdepth}。
\item 第一个和最后一个标记放在 \cs{splitfirstmark} 和 \cs{splitbotmark}寄存器中。
\end{itemize}

%The remainder of the \cs{vsplit} operation is a box where
%\begin{itemize} \item all discardables have been removed
%      from the top;
%\item glue of size \csidx{splittopskip} has been inserted on top;
%      if the box being split was box~255, it
%      already had \cs{topskip} glue on top;
%\item its depth has been forced to be at most \cs{splitmaxdepth}.
%\end{itemize}
而经过 \cs{vsplit} 操作后盒子的剩余部分是这样的:
\begin{itemize}
\item 顶部的所有可弃项都被删除;
\item 大小为 \csidx{splittopskip} 的粘连被插入到顶部;
      如果所分割的是第~255~号盒子,它的顶部已经有了 \cs{topskip} 粘连;
\item 其深度不得超过 \cs{splitmaxdepth}。
\end{itemize}

%The bottom of the original box is always a valid breakpoint
%for the \cs{vsplit} operation. If this breakpoint is taken,
%the remainder box register is void. The extracted box
%can be empty; it is only void if the original box
%was void, or not a vertical box.
原始盒子的底部始终是 \cs{vsplit} 操作的合适断点。
如果选择它为断点,剩余的盒子寄存器就是空的。提取出的盒子也可以是空的;
它是空的当且仅当原始盒子是空的或者并非竖直盒子。

%Typically, the \cs{vsplit} operation is used to split off part
%of \cs{box255}. By setting \cs{splitmaxdepth} equal to \cs{boxmaxdepth}
%the result is something that could have been made by \TeX's page
%builder. After pruning the top of \cs{box255}, the 
%mark registers \cs{firstmark} and \cs{botmark} contain the first
%and last marks on the remainder of box~255.
%See the next chapter for more information on marks.
通常,\cs{vsplit} 被用于从 \cs{box255} 分割出一部分。
通过设定 \cs{splitmaxdepth} 等于 \cs{boxmaxdepth},
得到的结果就和 \TeX\ 的页面构建器的一样。
在修剪 \cs{box255} 顶部之后,标记寄存器 \cs{firstmark} 和 \cs{botmark}
就包含 \cs{box255} 剩余部分的第一个和最后一个标记。
见下一章关于标记的更多信息。

%%\point Examples of page breaking
%\section{Examples of page breaking}
%\point Examples of page breaking
\section{分页的例子}

%%\spoint Filling up a page
%\subsection{Filling up a page}
%\spoint Filling up a page
\subsection{填满页面}

%Suppose a certain vertical box is too large
%to fit on the remainder of the page.
%Then
%\begin{verbatim}
%\vfil\vbox{ ... }
%\end{verbatim}
%is the wrong way
%to fill up the page and push the box to the next.
%\TeX\ can only break at the start of the glue, and
%the \cs{vfil} is discarded after the break: the result
%is an underfull, or at least horribly stretched, page.
%On the other hand,
%\begin{verbatim}
%\vfil\penalty0 % or any other value
%\vbox{ ... }
%\end{verbatim}
%is the correct way: \TeX\ will break
%at the penalty, and the page will be filled.
假设某个竖直盒子太大以致在页面剩余部分放不下。
要将页面填满并将该盒子放到下一页,使用
\begin{verbatim}
\vfil\vbox{ ... }
\end{verbatim}
是错误的。这样 \TeX\ 只会在 \cs{vfil} 粘连开始处分页,
而分页后 \cs{vfil} 将会被丢弃:结果将得到一个未满页面,
或者至少是一个有伸长太多的糟糕页面。另一方面,
\begin{verbatim}
\vfil\penalty0 % or any other value
\vbox{ ... }
\end{verbatim}
就是正确的做法:\TeX\ 将在惩罚处分页,而该页面将被填满。

%%\spoint Determining the breakpoint
%\subsection{Determining the breakpoint}
%\spoint Determining the breakpoint
\subsection{确定断点}

%In the following examples the \cs{vsplit} operation is
%used, which has the same
%mechanism as page breaking.
在接下来的例子中我们使用与分页机制相同的 \cs{vsplit} 操作。

%Let the macros and
%parameter settings
%\begin{verbatim}
%\offinterlineskip \showboxdepth=1
%\def\High{\hbox{\vrule height5pt}}
%\def\HighAndDeep{\hbox{\vrule height2.5pt depth2.5pt}}
%\end{verbatim}
%be given.
假设已经给出下面的宏和参数设定:
\begin{verbatim}
\offinterlineskip \showboxdepth=1
\def\High{\hbox{\vrule height5pt}}
\def\HighAndDeep{\hbox{\vrule height2.5pt depth2.5pt}}
\end{verbatim}

%First let us consider
%an example where a vertical list is simply stretched
%in order to reach a break point.
%\begin{verbatim}
%\splitmaxdepth=4pt 
%\setbox1=\vbox{\High \vfil \HighAndDeep}
%\setbox2=\vsplit1 to 9pt
%\end{verbatim}
%gives
%\begin{verbatim}
%> \box2=
%\vbox(9.0+2.5)x0.4, glue set 1.5fil
%.\hbox(5.0+0.0)x0.4 []
%.\glue 0.0 plus 1.0fil
%.\glue(\lineskip) 0.0
%.\hbox(2.5+2.5)x0.4 []
%\end{verbatim}
%The two boxes together have a height of \n{7.5pt},
%so the glue has to stretch~\n{1.5pt}.
首先我们考虑伸长竖直列以到达断点的例子。 
\begin{verbatim}
\splitmaxdepth=4pt 
\setbox1=\vbox{\High \vfil \HighAndDeep}
\setbox2=\vsplit1 to 9pt
\end{verbatim}
给出
\begin{verbatim}
> \box2=
\vbox(9.0+2.5)x0.4, glue set 1.5fil
.\hbox(5.0+0.0)x0.4 []
.\glue 0.0 plus 1.0fil
.\glue(\lineskip) 0.0
.\hbox(2.5+2.5)x0.4 []
\end{verbatim}
这两个盒子合起来的高度为 \n{7.5pt},因此粘连需要伸长~\n{1.5pt}。

%Next, we decrease the allowed depth of the resulting list.
%\begin{verbatim}
%\splitmaxdepth=2pt 
%\setbox1=\vbox{\High \vfil \HighAndDeep}
%\setbox2=\vsplit1 to 9pt
%\end{verbatim}
%gives
%\begin{verbatim}
%> \box2=
%\vbox(9.0+2.0)x0.4, glue set 1.0fil
%.\hbox(5.0+0.0)x0.4 []
%.\glue 0.0 plus 1.0fil
%.\glue(\lineskip) 0.0
%.\hbox(2.5+2.5)x0.4 []
%\end{verbatim}
%The reference point is moved down half a point,
%and the stretch is correspondingly diminished,
%\alt
%but this motion cannot lead to a larger dimension
%than was specified.
其次,我们减少结果列表所允许的深度值。
\begin{verbatim}
\splitmaxdepth=2pt 
\setbox1=\vbox{\High \vfil \HighAndDeep}
\setbox2=\vsplit1 to 9pt
\end{verbatim}
给出
\begin{verbatim}
> \box2=
\vbox(9.0+2.0)x0.4, glue set 1.0fil
.\hbox(5.0+0.0)x0.4 []
.\glue 0.0 plus 1.0fil
.\glue(\lineskip) 0.0
.\hbox(2.5+2.5)x0.4 []
\end{verbatim}
此时基准点被下移 \n{0.5pt},而伸长量也相应减少,
\alt
但是结果列表的尺寸不会比所指定的大。

%As an example of this,
%\alt
%consider the sequence
%\begin{verbatim}
%\splitmaxdepth=3pt 
%\setbox1=\vbox{\High \kern1.5pt \HighAndDeep}
%\setbox2=\vsplit1 to 9pt
%\end{verbatim}
%This gives a box exactly 9 points high and 2.5 points deep.
%Setting \verb>\splitmaxdepth=2pt> does not increase
%the height by half a point; instead, an underfull box
%results because an earlier break is taken.
作为这个事实的例子,
\alt
再考虑下面的例子
\begin{verbatim}
\splitmaxdepth=3pt 
\setbox1=\vbox{\High \kern1.5pt \HighAndDeep}
\setbox2=\vsplit1 to 9pt
\end{verbatim}
这将会给出一个恰好 \n{9pt} 高 \n{2.5pt} 深的盒子:
\begin{verbatim}
> \box2=
\vbox(9.0+2.5)x0.4
.\hbox(5.0+0.0)x0.4 []
.\kern 1.5
.\glue(\lineskip) 0.0
.\hbox(2.5+2.5)x0.4 []
\end{verbatim}
若改为 \verb>\splitmaxdepth=2pt>,将不会让高度增加 \n{0.5pt};
反而由于断点提前而得到一个未满盒子:
\begin{verbatim}
> \box2=
\vbox(9.0+0.0)x0.4
.\hbox(5.0+0.0)x0.4 []
\end{verbatim}

%Sometimes the timing of actions is important.
%\TeX\ first locates a breakpoint that will lead
%to the requested height, then checks whether accommodating
%the \cs{maxdepth} or \cs{splitmaxdepth} will not
%violate that height.
有时候分割的时机也是重要的。
\TeX\ 首先寻找能得到所需高度的断点,然后看依照 \cs{maxdepth}
或 \cs{splitmaxdepth} 调节深度是否会超出规定高度。

%Consider an example of this timing:
%\alt
%in
%\begin{verbatim}
%\splitmaxdepth=4pt 
%\setbox1=\vbox{\High \vfil \HighAndDeep}
%\setbox2=\vsplit1 to 7pt
%\end{verbatim}
%the result is {\italic not\/} a box
%of 7 points high and 3 points deep. Instead,
%\begin{verbatim}
%> \box2=
%\vbox(7.0+0.0)x0.4
%.\hbox(5.0+0.0)x0.4 []
%\end{verbatim}
%which is an underfull box.
考虑一个与此时机有关的例子:
\alt
这段代码
\begin{verbatim}
\splitmaxdepth=4pt 
\setbox1=\vbox{\High \vfil \HighAndDeep}
\setbox2=\vsplit1 to 7pt
\end{verbatim}
所得结果{\italic 并非\/}一个 \n{7pt} 高 \n{3pt} 深的盒子,
而是一个未满盒子:
\begin{verbatim}
> \box2=
\vbox(7.0+0.0)x0.4
.\hbox(5.0+0.0)x0.4 []
\end{verbatim}

%%\spoint[par:page:build] The page builder after a paragraph
%\subsection{The page builder after a paragraph}
%\label{par:page:build}
%\spoint[par:page:build] The page builder after a paragraph
\subsection{段落后的页面构建}
\label{par:page:build}

%After a paragraph, the page builder moves material
%to the current page, but it does not decide whether a breakpoint
%has been found yet.
在段落结束后,页面构建器将素材移动到当前页面,
但它并不决定断点是否已经找到。

%\begin{example}
%\begin{verbatim}
%\output{\interrupt \plainoutput}% show when you're active
%\def\nl{\hfil\break}\vsize=22pt % make pages of two lines
%a\nl b\nl c\par \showlists      % make a 3-line paragraph
%\end{verbatim}
%will report
%\begin{verbatim}
%### current page:
%[...]
%total height 34.0
% goal height 22.0
%prevdepth 0.0, prevgraf 3 lines
%\end{verbatim}
%Even though more than enough
%material has been gathered, \cs{output} is only invoked
%when the next paragraph starts: typing a \n d gives
%\begin{verbatim}
%! Undefined control sequence.
%<output> {\interrupt 
%                     \plainoutput }
%<to be read again> 
%                   d
%\end{verbatim}
%when \cs{output} is inserted after \cs{everypar}.
%\end{example}
\begin{example}
\begin{verbatim}
\output{\interrupt \plainoutput}% show when you're active
\def\nl{\hfil\break}\vsize=22pt % make pages of two lines
a\nl b\nl c\par \showlists      % make a 3-line paragraph
\end{verbatim}
将报告
\begin{verbatim}
### current page:
[...]
total height 34.0
 goal height 22.0
prevdepth 0.0, prevgraf 3 lines
\end{verbatim}
纵使已经收集到过多的素材,\cs{output} 还是仅会在下个段落开始时才执行:
当 \cs{output} 被插入到 \cs{everypar} 后,键入 \n d 将给出
\begin{verbatim}
! Undefined control sequence.
<output> {\interrupt 
                     \plainoutput }
<to be read again> 
                   d
\end{verbatim}
\end{example}

%\index{page!breaking|)}
\index{page!breaking|)}

%\endofchapter
\endofchapter

\end{document}
