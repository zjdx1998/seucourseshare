% -*- coding: utf-8 -*-
\documentclass{book}

% -*- coding: utf-8 -*-

\usepackage[b5paper,text={5in,8in},centering]{geometry}
\usepackage{amsmath}
\usepackage[heading = false, scheme = plain]{ctex}
% \usepackage[CJKchecksingle]{xeCJK}
\setmainfont[Mapping=tex-text]{TeX Gyre Schola}
%\setsansfont{URW Gothic L Book}
%\setmonofont{Nimbus Mono L}
% \setCJKmainfont[BoldFont=FandolHei,ItalicFont=FandolKai]{FandolSong}
% \setCJKsansfont{FandolHei}
% \setCJKmonofont{FandolFang}
\xeCJKsetup{PunctStyle = kaiming}

\linespread{1.25}
\setlength{\parindent}{2em}
\setlength{\parskip}{0.5ex}
\usepackage{indentfirst}

\usepackage{xcolor}
\definecolor{myblue}{rgb}{0,0.2,0.6}

\usepackage{titlesec}
\titleformat{\chapter}
    {\normalfont\Huge\sffamily\color{myblue}}
    {第\thechapter 章}
    {1em}
    {}
%\titlespacing{\chapter}{0pt}{50pt}{40pt}
\titleformat{\section}
    {\normalfont\Large\sffamily\color{myblue}}
    {\thesection}
    {1em}
    {}
%\titlespacing{\section}{0pt}{3.5ex plus 1ex minus .2ex}{2.3ex plus .2ex}
\titleformat{\subsection}
    {\normalfont\large\sffamily\color{myblue}}
    {\thesubsection}
    {1em}
    {}
%\titlespacing{\subsection}{0pt}{3.25ex plus 1ex minus .2ex}{1.5ex plus .2ex}
%
\newpagestyle{special}[\small\sffamily]{
  \headrule
  \sethead[\usepage][][\chaptertitle]
  {\chaptertitle}{}{\usepage}}
\newpagestyle{main}[\small\sffamily]{
  \headrule
  \sethead[\usepage][][第\thechapter 章\quad\chaptertitle]
  {\thesection\quad\sectiontitle}{}{\usepage}}

\usepackage{titletoc}
%\setcounter{tocdepth}{1}
%\titlecontents{标题层次}[左间距]{上间距和整体格式}{标题序号}{标题内容}{指引线和页码}[下间距]
\titlecontents{chapter}[1.5em]{\vspace{.5em}\bfseries\sffamily}{\color{myblue}\contentslabel{1.5em}}{}
    {\titlerule*[20pt]{$\cdot$}\contentspage}[]
\titlecontents{section}[4.5em]{\sffamily}{\color{myblue}\contentslabel{3em}}{}
    {\titlerule*[20pt]{$\cdot$}\contentspage}[]
%\titlecontents{subsection}[8.5em]{\sffamily}{\contentslabel{4em}}{}
%    {\titlerule*[20pt]{$\cdot$}\contentspage}

\usepackage{enumitem}
\setlist{topsep=2pt,itemsep=2pt,parsep=1pt,leftmargin=\parindent}

\usepackage{fancyvrb}
\DefineVerbatimEnvironment{verbatim}{Verbatim}
  {xleftmargin=2em,baselinestretch=1,formatcom=\color{teal}\upshape}

\usepackage{etoolbox}
\makeatletter
\preto{\FV@ListVSpace}{\topsep=2pt \partopsep=0pt }
\makeatother

\usepackage[colorlinks,plainpages,pagebackref]{hyperref}
\hypersetup{
   pdfstartview={FitH},
   citecolor=teal,
   linkcolor=myblue,
   urlcolor=black,
   bookmarksnumbered
}

\usepackage{comment,makeidx,multicol}

%\usepackage{german}
%% german
%\righthyphenmin=3
%\mdqoff
%\captionsenglish
\usepackage[english]{babel}
{\catcode`"=13 \gdef"#1{\ifx#1"\discretionary{}{}{}\fi\relax}}
\def\mdqon{\catcode`"=13\relax}
\def\mdqoff{\catcode`"=12\relax}
\makeindex
\hyphenation{ex-em-pli-fies}

\newdimen\tempdima \newdimen\tempdimb

% these are fine
\def\handbreak{\\ \message{^^JManual break!!!!^^J}}
\def\nl{\protect\\}\def\n#1{{\tt #1}}
\protected\def\cs#1{\texttt{\textbackslash#1}}
\pdfstringdefDisableCommands{\def\cs#1{\textbackslash#1}}
\let\csc\cs
\def\lb{{\tt\char`\{}}\def\rb{{\tt\char`\}}}
\def\gr#1{\texorpdfstring{$\langle$#1$\rangle$}{<#1>}} %\def\gr#1{$\langle$#1$\rangle$}
\def\marg#1{{\tt \{}#1{\tt \}}}
\def\oarg#1{{\tt [}#1{\tt }}
\def\key#1{{\tt#1}}
\def\alt{}\def\altt{}%this way in manstijl
\def\ldash{\unskip\ ——\nobreak\ \ignorespaces}
\def\rdash{\unskip\nobreak\ ——\ \ignorespaces}
% check these
\def\hex{{\tt"}}
\def\ascii{{\sc ascii}}
\def\ebcdic{{\sc ebcdic}}
\def\IniTeX{Ini\TeX}\def\LamsTeX{LAMS\TeX}\def\VirTeX{Vir\TeX}
\def\AmsTeX{Ams\TeX}
\def\TeXbook{the \TeX\ book}\def\web{{\sc web}}
% needs major thinking
\newenvironment{myquote}{\list{}{%
    \topsep=2pt \partopsep=0pt%
    \leftmargin=\parindent \rightmargin=\parindent
    }\item[]}{\endlist}
\newenvironment{disp}{\begin{myquote}}{\end{myquote}}
\newenvironment{Disp}{\begin{myquote}}{\end{myquote}}
\newenvironment{tdisp}{\begin{myquote}}{\end{myquote}}
\newenvironment{example}{\begin{myquote}\noindent\itshape 例子:}{\end{myquote}}
\newenvironment{inventory}{\begin{description}\raggedright}{\end{description}}
\newenvironment{glossinventory}{\begin{description}}{\end{description}}
\def\gram#1{\gr{#1}}%???
\def\meta{\gr}% alias
%
% index
%
\def\indexterm#1{\emph{#1}\index{#1}}
\def\indextermsub#1#2{\emph{#1 #2}\index{#1!#2}}
\def\indextermbus#1#2{\emph{#1 #2}\index{#2!#1}}
\def\cindextermsub#1#2{\emph{#1#2}\index{#1!#1#2}}
\def\cindextermbus#1#2{\emph{#1#2}\index{#2!#1#2}}
\def\term#1\par{\index{#1}}
\def\howto#1\par{}
\def\cstoidx#1\par{\index{#1@\cs{#1}@}}
\def\thecstoidx#1\par{\index{#1@\protect\csname #1\endcsname}}
\def\thecstoidxsub#1#2{\index{#1, #2@\protect\csname #1\endcsname, #2}\ignorespaces}
\def\csterm#1\par{\cstoidx #1\par\cs{#1}}
\def\csidx#1{\cstoidx #1\par\cs{#1}}

\def\tmc{\tracingmacros=2 \tracingcommands\tracingmacros}

%%%%%%%%%%%%%%%%%%%
\makeatletter
\def\snugbox{\hbox\bgroup\setbox\z@\vbox\bgroup
    \leftskip\z@
    \bgroup\aftergroup\make@snug
    \let\next=}
\def\make@snug{\par\sn@gify\egroup \box\z@\egroup}
\def\sn@gify
   {\skip\z@=\lastskip \unskip
    \advance\skip\z@\lastskip \unskip
    \unpenalty
    \setbox\z@\lastbox
    \ifvoid\z@ \nointerlineskip \else {\sn@gify} \fi
    \hbox{\unhbox\z@}\nointerlineskip
    \vskip\skip\z@
    }

\newdimen\fbh \fbh=60pt % dimension for easy scaling:
\newdimen\fbw \fbw=60pt % height and width of character box

\newdimen\dh \newdimen\dw % height and width of current character box
\newdimen\lh % height of previous character box
\newdimen\lw \lw=.4pt % line weight, instead of default .4pt

\def\hdotfill{\noindent
    \leaders\hbox{\vrule width 1pt height\lw
                  \kern4pt
                  \vrule width.5pt height\lw}\hfill\hbox{}
    \par}
\def\hlinefill{\noindent
    \leaders\hbox{\vrule width 5.5pt height\lw         }\hfill\hbox{}
    \par}
\def\stippel{$\qquad\qquad\qquad\qquad$}
\makeatother
%%%%%%%%%%%%%%%%%%%

%\def\SansSerif{\Typeface:macHelvetica }
%\def\SerifFont{\Typeface:macTimes }
%\def\SansSerif{\Typeface:bsGillSans }
%\def\SerifFont{\Typeface:bsBaskerville }
\let\SansSerif\relax \def\italic{\it}
\let\SerifFont\relax \def\MainFont{\rm}
\let\SansSerif\relax
\let\SerifFont\relax
\let\PopIndentLevel\relax \let\PushIndentLevel\relax
\let\ToVerso\relax \let\ToRecto\relax

%\def\stop@command@suffix{stop}
%\let\PopListLevel\PopIndentLevel
%\let\FlushRight\relax
%\let\flushright\FlushRight
%\let\SetListIndent\LevelIndent
%\def\awp{\ifhmode\vadjust{\penalty-10000 }\else
%    \penalty-10000 \fi}
\let\awp\relax
\let\PopIndentLevel\relax \let\PopListLevel\relax

\showboxdepth=-1

%\input figs
\def\endofchapter{\vfill\noindent}

\newcommand{\liamfnote}[1]{\footnote{译注(Liam0205):#1}}
\newcommand{\cstate}[1]{状态 \textit{#1}}

\setcounter{chapter}{24}

\begin{document}

%\chapter{Alignment}\label{align}
%\index{alignments|(}
\chapter{Alignment}\label{align}
\index{alignments|(}

%\TeX\ provides a general alignment mechanism for making \indexterm{tables}.
\TeX\ provides a general alignment mechanism for making \indexterm{tables}.

%\label{cschap:halign}\label{cschap:valign}\label{cschap:omit}\label{cschap:span}\label{cschap:multispan}\label{cschap:tabskip}\label{cschap:noalign}\label{cschap:cr}\label{cschap:crcr}\label{cschap:everycr}\label{cschap:centering}\label{cschap:hideskip}\label{cschap:hidewidth}
%\begin{inventory}
%\item [\cs{halign}] 
%      Horizontal alignment.
\label{cschap:halign}\label{cschap:valign}\label{cschap:omit}\label{cschap:span}\label{cschap:multispan}\label{cschap:tabskip}\label{cschap:noalign}\label{cschap:cr}\label{cschap:crcr}\label{cschap:everycr}\label{cschap:centering}\label{cschap:hideskip}\label{cschap:hidewidth}
\begin{inventory}
\item [\cs{halign}] 
      Horizontal alignment.

%\item [\cs{valign}] 
%      Vertical alignment.   
\item [\cs{valign}] 
      Vertical alignment.   

%\item [\cs{omit}] 
%      Omit the template for one alignment entry.
\item [\cs{omit}] 
      Omit the template for one alignment entry.

%\item [\cs{span}] 
%      Join two adjacent alignment entries.
\item [\cs{span}] 
      Join two adjacent alignment entries.

%\item [\cs{multispan}] 
%      Macro to join a number of adjacent alignment entries.
\item [\cs{multispan}] 
      Macro to join a number of adjacent alignment entries.

%\item [\cs{tabskip}] 
%      Amount of glue in between columns (rows)
%      of an \cs{halign} (\cs{valign}). 
\item [\cs{tabskip}] 
      Amount of glue in between columns (rows)
      of an \cs{halign} (\cs{valign}). 

%\item [\cs{noalign}] 
%      Specify  vertical (horizontal)
%      material   to be placed in between rows (columns) of
%      an \cs{halign} (\cs{valign}).
\item [\cs{noalign}] 
      Specify  vertical (horizontal)
      material   to be placed in between rows (columns) of
      an \cs{halign} (\cs{valign}).

%\item [\cs{cr}]
%      Terminate an alignment line.
\item [\cs{cr}]
      Terminate an alignment line.

%\item [\cs{crcr}] 
%      Terminate an alignment line if it has 
%      not already been terminated by~\cs{cr}.
\item [\cs{crcr}] 
      Terminate an alignment line if it has 
      not already been terminated by~\cs{cr}.

%\item [\cs{everycr}] 
%      Token list inserted after every \cs{cr} or non-redundant 
%      \cs{crcr}.
\item [\cs{everycr}] 
      Token list inserted after every \cs{cr} or non-redundant 
      \cs{crcr}.

%\item [\cs{centering}] 
%      Glue register in plain \TeX\ for centring
%      \cs{eqalign} and \cs{eqalignno}.
%      Value: \n{0pt plus 1000pt minus 1000pt}
\item [\cs{centering}] 
      Glue register in plain \TeX\ for centring
      \cs{eqalign} and \cs{eqalignno}.
      Value: \n{0pt plus 1000pt minus 1000pt}

%\item [\cs{hideskip}]
%      Glue register in plain \TeX\ to make alignment entries invisible.
%      Value: \n{-1000pt plus 1fill}
\item [\cs{hideskip}]
      Glue register in plain \TeX\ to make alignment entries invisible.
      Value: \n{-1000pt plus 1fill}

%\item [\cs{hidewidth}]
%      Macro to make preceding or following entry invisible.
\item [\cs{hidewidth}]
      Macro to make preceding or following entry invisible.

%\end{inventory}
\end{inventory}

%%\point Introduction
%\section{Introduction}
%\point Introduction
\section{Introduction}

%\TeX\ has a sophisticated alignment mechanism, based on 
%templates, with one template entry per column or row.
%The templates may contain any common elements
%of the table entries, and in general they contain
%instructions for typesetting the entries.
%\TeX\ first calculates widths (for \cs{halign}) or heights
%(for \cs{valign}) of all entries;
%then it typesets the whole alignment using in each column (row)
%the maximum width (height) of entries in that column (row).
\TeX\ has a sophisticated alignment mechanism, based on 
templates, with one template entry per column or row.
The templates may contain any common elements
of the table entries, and in general they contain
instructions for typesetting the entries.
\TeX\ first calculates widths (for \cs{halign}) or heights
(for \cs{valign}) of all entries;
then it typesets the whole alignment using in each column (row)
the maximum width (height) of entries in that column (row).

%%\point Horizontal and vertical alignment
%\section{Horizontal and vertical alignment}
%\point Horizontal and vertical alignment
\section{Horizontal and vertical alignment}

%The two alignment commands in \TeX\ are
%\cstoidx halign\par\cstoidx valign\par
%\begin{disp}\cs{halign}\gr{box specification}\lb\gr{alignment material}\rb
%\end{disp} for horizontal alignment of columns, and
%\begin{disp}\cs{valign}\gr{box specification}\lb\gr{alignment material}\rb
%\end{disp} for vertical alignment of rows.
%\cs{halign} is a \gr{vertical command}, and
%\cs{valign} is a \gr{horizontal command}.
The two alignment commands in \TeX\ are
\cstoidx halign\par\cstoidx valign\par
\begin{disp}\cs{halign}\gr{box specification}\lb\gr{alignment material}\rb
\end{disp} for horizontal alignment of columns, and
\begin{disp}\cs{valign}\gr{box specification}\lb\gr{alignment material}\rb
\end{disp} for vertical alignment of rows.
\cs{halign} is a \gr{vertical command}, and
\cs{valign} is a \gr{horizontal command}.

%The braces induce a new level of grouping; they can be
%implicit.
The braces induce a new level of grouping; they can be
implicit.

%The discussion below will mostly focus on horizontal
%alignments, but, replacing `column' by `row' and vice versa,
%it applies to vertical alignments too.
The discussion below will mostly focus on horizontal
alignments, but, replacing `column' by `row' and vice versa,
it applies to vertical alignments too.

%%\spoint Horizontal alignments: \cs{halign}
%\subsection{Horizontal alignments: \cs{halign}}
%\spoint Horizontal alignments: \cs{halign}
\subsection{Horizontal alignments: \cs{halign}}

%A \indexterm{horizontal alignment}
%yields a list of horizontal boxes, the rows,
%which are placed on the surrounding vertical list.
%The page builder is exercised after the alignment rows have been
%added to  the vertical list.
%The value of \cs{prevdepth} that holds before the alignment
%is used for the baselineskip of the first row,
%and after the alignment \cs{prevdepth} is set to a value based
%on the last row.
A \indexterm{horizontal alignment}
yields a list of horizontal boxes, the rows,
which are placed on the surrounding vertical list.
The page builder is exercised after the alignment rows have been
added to  the vertical list.
The value of \cs{prevdepth} that holds before the alignment
is used for the baselineskip of the first row,
and after the alignment \cs{prevdepth} is set to a value based
on the last row.

%Each entry is processed in a group of its own,
%in restricted horizontal mode.
Each entry is processed in a group of its own,
in restricted horizontal mode.

%A special type of horizontal alignment exists: the
%\indexterm{display alignment},
%specified as
%\begin{disp}\n{\$\$}\gr{assignments}\cs{halign}\gr{box specification}\lb\n{...}\rb
%     \gr{assignments}\n{\$\$}\end{disp}
%Such an alignment is shifted by \cs{displayindent} (see
%Chapter~\ref{displaymath}) and surrounded by\handbreak
%\cs{abovedisplayskip} and \cs{belowdisplayskip} glue.
A special type of horizontal alignment exists: the
\indexterm{display alignment},
specified as
\begin{disp}\n{\$\$}\gr{assignments}\cs{halign}\gr{box specification}\lb\n{...}\rb
     \gr{assignments}\n{\$\$}\end{disp}
Such an alignment is shifted by \cs{displayindent} (see
Chapter~\ref{displaymath}) and surrounded by\handbreak
\cs{abovedisplayskip} and \cs{belowdisplayskip} glue.

%%\spoint Vertical alignments: \cs{valign}
%\subsection{Vertical alignments: \cs{valign}}
%\spoint Vertical alignments: \cs{valign}
\subsection{Vertical alignments: \cs{valign}}

%A \indexterm{vertical alignment}
%can be considered as a  `rotated' horizontal alignments:
%they are placed on the surrounding horizontal lists,
%and yield a row of columns. The \cs{spacefactor} value
%is treated the same way as the \cs{prevdepth} for horizontal
%alignments: the value current before the alignment is used
%for the first column, and the value reached after the last column
%is used after the alignment. In between columns the \cs{spacefactor}
%value is~1000.
A \indexterm{vertical alignment}
can be considered as a  `rotated' horizontal alignments:
they are placed on the surrounding horizontal lists,
and yield a row of columns. The \cs{spacefactor} value
is treated the same way as the \cs{prevdepth} for horizontal
alignments: the value current before the alignment is used
for the first column, and the value reached after the last column
is used after the alignment. In between columns the \cs{spacefactor}
value is~1000.

%Each entry is in a group of its own, and it is processed
%in internal vertical mode.
Each entry is in a group of its own, and it is processed
in internal vertical mode.

%%\spoint Material between the lines: \cs{noalign}
%\subsection{Material between the lines: \cs{noalign}}
%\spoint Material between the lines: \cs{noalign}
\subsection{Material between the lines: \cs{noalign}}

%Material that has to be contained in the alignment, but 
%should not be treated as an entry or series of entries,
%\cstoidx noalign\par
%can be given by 
%\begin{disp}\cs{noalign}\gr{filler}\lb\gr{vertical mode material}\rb
%\end{disp} for horizontal alignments, and
%\begin{disp}\cs{noalign}\gr{filler}\lb\gr{horizontal mode material}\rb
%\end{disp} for vertical alignments.
Material that has to be contained in the alignment, but 
should not be treated as an entry or series of entries,
\cstoidx noalign\par
can be given by 
\begin{disp}\cs{noalign}\gr{filler}\lb\gr{vertical mode material}\rb
\end{disp} for horizontal alignments, and
\begin{disp}\cs{noalign}\gr{filler}\lb\gr{horizontal mode material}\rb
\end{disp} for vertical alignments.

%Examples are
%\begin{verbatim}
%\noalign{\hrule}
%\end{verbatim}
%for drawing a horizontal rule
%between two lines of an \cs{halign},
%and
%\begin{verbatim}
%\noalign{\penalty100}
%\end{verbatim}
%for discouraging a page break (or line break) in
%between two rows (columns) of an \cs{halign} (\cs{valign}).
Examples are
\begin{verbatim}
\noalign{\hrule}
\end{verbatim}
for drawing a horizontal rule
between two lines of an \cs{halign},
and
\begin{verbatim}
\noalign{\penalty100}
\end{verbatim}
for discouraging a page break (or line break) in
between two rows (columns) of an \cs{halign} (\cs{valign}).

%%\spoint Size of the alignment
%\subsection{Size of the alignment}
%\spoint Size of the alignment
\subsection{Size of the alignment}

%The \gr{box specification} can be used to give the alignment
%a predetermined size: for instance
%\begin{verbatim}
%\halign to \hsize{ ... }
%\end{verbatim}
%Glue contained in the entries of the alignment has no role in this;
%any stretch or
%shrink required is taken from the \cs{tabskip} glue.
%This is explained below.
The \gr{box specification} can be used to give the alignment
a predetermined size: for instance
\begin{verbatim}
\halign to \hsize{ ... }
\end{verbatim}
Glue contained in the entries of the alignment has no role in this;
any stretch or
shrink required is taken from the \cs{tabskip} glue.
This is explained below.

%%\point The preamble
%\section{The preamble}
%\point The preamble
\section{The preamble}

%Each line in an alignment is terminated by \cs{cr};
%the first line is called the {\it template line}.
%It is of the form
%\begin{disp}\n{$u_1$\#$v_1$\&...\&$u_n$\#$v_n$}\cs{cr}\end{disp}
%where each $u_i$, $v_i$ is a (possibly empty) arbitrary sequence
%of tokens, and the template entries are separated by
%the 
%\indexterm{alignment tab}
%character (\n\&~in plain \TeX),
%that is, any character of category~4\index{category!4}.
Each line in an alignment is terminated by \cs{cr};
the first line is called the {\it template line}.
It is of the form
\begin{disp}\n{$u_1$\#$v_1$\&...\&$u_n$\#$v_n$}\cs{cr}\end{disp}
where each $u_i$, $v_i$ is a (possibly empty) arbitrary sequence
of tokens, and the template entries are separated by
the 
\indexterm{alignment tab}
character (\n\&~in plain \TeX),
that is, any character of category~4\index{category!4}.

%A $u_i$\n\#$v_i$ sequence is the template that will be
%used for the $i\,$th column: whatever sequence $\alpha_i$
%the user specifies
%as the entry for that column will be inserted at the
%parameter character. The sequence $u_i\alpha_iv_i$ is
%then processed to obtain the actual entry for the $i\,$th
%column on the current line. See below for more details.
A $u_i$\n\#$v_i$ sequence is the template that will be
used for the $i\,$th column: whatever sequence $\alpha_i$
the user specifies
as the entry for that column will be inserted at the
parameter character. The sequence $u_i\alpha_iv_i$ is
then processed to obtain the actual entry for the $i\,$th
column on the current line. See below for more details.

%The length $n$ of the template line need
%not be equal to the actual number of columns in the alignment:
%the template is used only for as many items as are specified
%on a line. Consider as an example
%\begin{verbatim}
%\halign{a#&b#&c#\cr 1&2\cr 1\cr}
%\end{verbatim}
%which has a three-item template, but the rows have only
%one or two items. The output of this is
%\begin{disp}\leavevmode\vbox{\halign{a#&b#&c#\cr 1&2\cr 1\cr}}\end{disp}
The length $n$ of the template line need
not be equal to the actual number of columns in the alignment:
the template is used only for as many items as are specified
on a line. Consider as an example
\begin{verbatim}
\halign{a#&b#&c#\cr 1&2\cr 1\cr}
\end{verbatim}
which has a three-item template, but the rows have only
one or two items. The output of this is
\begin{disp}\leavevmode\vbox{\halign{a#&b#&c#\cr 1&2\cr 1\cr}}\end{disp}

%%\spoint Infinite preambles
%\subsection{Infinite preambles}
%\spoint Infinite preambles
\subsection{Infinite preambles}

%For the case where the number of columns is not known in advance,
%for instance if the alignment is to be used in a macro where
%the user will specify the columns, it is possible to
%specify that a trailing piece of the
%preamble can be repeated arbitrarily many times.
%By preceding it with \n\&, an entry can be marked as the
%start of this repeatable part of the preamble.
%See the example of \cs{matrix} below.
For the case where the number of columns is not known in advance,
for instance if the alignment is to be used in a macro where
the user will specify the columns, it is possible to
specify that a trailing piece of the
preamble can be repeated arbitrarily many times.
By preceding it with \n\&, an entry can be marked as the
start of this repeatable part of the preamble.
See the example of \cs{matrix} below.

%When the whole preamble is to be repeated, there will be
%an alignment tab character at the start of the first entry:
%\begin{verbatim}
%\halign{& ... & ... \cr ... }
%\end{verbatim}
%If a starting portion of the preamble is to be exempted from
%repetition, a double alignment tab will occur:
%\begin{verbatim}
%\halign{ ... & ... & ... && ... & ... \cr ... }
%\end{verbatim}
When the whole preamble is to be repeated, there will be
an alignment tab character at the start of the first entry:
\begin{verbatim}
\halign{& ... & ... \cr ... }
\end{verbatim}
If a starting portion of the preamble is to be exempted from
repetition, a double alignment tab will occur:
\begin{verbatim}
\halign{ ... & ... & ... && ... & ... \cr ... }
\end{verbatim}

%The  repeatable part need not be used an integral
%number of times. The alignment rows can end at any time;
%the rest of the preamble is then not used.
The  repeatable part need not be used an integral
number of times. The alignment rows can end at any time;
the rest of the preamble is then not used.

%%\spoint Brace counting in preambles
%\subsection{Brace counting in preambles}
%\spoint Brace counting in preambles
\subsection{Brace counting in preambles}

%Alignments may appear inside alignments, so \TeX\ uses the
%following rule to determine to which alignment
%an \n\&  or \cs{cr} control sequence belongs:
%\begin{disp} All tab characters and \cs{cr} tokens of an alignment
%      should be on the same level of grouping.\end{disp}
%From this it follows that tab characters and \cs{cr} tokens
%can appear inside an entry if they are nested in braces.
%This makes it possible to have nested alignments.
Alignments may appear inside alignments, so \TeX\ uses the
following rule to determine to which alignment
an \n\&  or \cs{cr} control sequence belongs:
\begin{disp} All tab characters and \cs{cr} tokens of an alignment
      should be on the same level of grouping.\end{disp}
From this it follows that tab characters and \cs{cr} tokens
can appear inside an entry if they are nested in braces.
This makes it possible to have nested alignments.

%%\spoint Expansion in the preamble
%\subsection{Expansion in the preamble}
%\spoint Expansion in the preamble
\subsection{Expansion in the preamble}

%All tokens in the preamble \ldash apart from the tab characters \rdash
%are stored for insertion in the entries of the alignment,
%but a token preceded by \csidx{span} is expanded while
%the preamble is scanned. See below for the function of
%\cs{span} in the rest of the alignment.
All tokens in the preamble \ldash apart from the tab characters \rdash
are stored for insertion in the entries of the alignment,
but a token preceded by \csidx{span} is expanded while
the preamble is scanned. See below for the function of
\cs{span} in the rest of the alignment.

%%\spoint \cs{tabskip}
%\subsection{\cs{tabskip}}
%\spoint \cs{tabskip}
\subsection{\cs{tabskip}}

%Entries in an alignment are set to take the width of the
%largest element in their column.
%Glue for separating columns can be specified by assigning
%to \csidx{tabskip}.
%\altt
%\TeX\ inserts this glue in
%between each pair of columns, and before the first and after the
%last column.
Entries in an alignment are set to take the width of the
largest element in their column.
Glue for separating columns can be specified by assigning
to \csidx{tabskip}.
\altt
\TeX\ inserts this glue in
between each pair of columns, and before the first and after the
last column.

%The value of \cs{tabskip} that holds outside the alignment is
%used before the first column, and after all subsequent columns,
%unless the preamble contains assignments to \cs{tabskip}.
%Any assignment to \cs{tabskip} is executed while \TeX\ is scanning
%the preamble; the value that holds when a tab character is
%reached will be used at that place in each row, and after all subsequent
%columns, unless further assignments occur.
%The value of \cs{tabskip} that holds when \cs{cr} is reached
%is used after the last column.
The value of \cs{tabskip} that holds outside the alignment is
used before the first column, and after all subsequent columns,
unless the preamble contains assignments to \cs{tabskip}.
Any assignment to \cs{tabskip} is executed while \TeX\ is scanning
the preamble; the value that holds when a tab character is
reached will be used at that place in each row, and after all subsequent
columns, unless further assignments occur.
The value of \cs{tabskip} that holds when \cs{cr} is reached
is used after the last column.

%Assignments to \cs{tabskip} in the preamble are local to the
%alignment, but not to the entry where they are given.
%These assignments are ordinary glue assignments:
%they remove any optional trailing space.
Assignments to \cs{tabskip} in the preamble are local to the
alignment, but not to the entry where they are given.
These assignments are ordinary glue assignments:
they remove any optional trailing space.

%As an example, in the following table there is no tabskip
%glue before the first and after the last column; 
%in between all columns there is stretchable tabskip.
%\begin{verbatim}
%\tabskip=0pt \halign to \hsize{
%    \vrule#\tabskip=0pt plus 1fil\strut& 
%    \hfil#\hfil& \vrule#& \hfil#\hfil& \vrule#& \hfil#\hfil& 
%    \tabskip=0pt\vrule#\cr
%  \noalign{\hrule}
%    &\multispan5\hfil Just a table\hfil&\cr
%  \noalign{\hrule}
%    &one&&two&&three&\cr &a&&b&&c&\cr
%  \noalign{\hrule}
%    }
%\end{verbatim}
%The result of this is
%\begin{disp}\PopListLevel
%\leavevmode\message{single indent and sufficient vertical}%
%\hbox{\leftskip0pt \rightskip0pt
% \vbox{\offinterlineskip
%\tabskip=0pt \halign to \hsize{\strut
%    \vrule#\tabskip=0pt plus 1fil\strut&
%    \hfil#\hfil& \vrule#& \hfil#\hfil& 
%                 \vrule#& \hfil#\hfil&
%    \tabskip=0pt\vrule#\cr
%    \noalign{\hrule}
%    &\multispan5\hfil Just a table\hfil&\cr
%    \noalign{\hrule}
%    &one&&two&&three&\cr
%    &a&&b&&c&\cr
%    \noalign{\hrule}
%    }}}\end{disp}
%All of the vertical rules
%of the table are in a separate column. This is the only way
%to get the space around the items to stretch.
As an example, in the following table there is no tabskip
glue before the first and after the last column; 
in between all columns there is stretchable tabskip.
\begin{verbatim}
\tabskip=0pt \halign to \hsize{
    \vrule#\tabskip=0pt plus 1fil\strut& 
    \hfil#\hfil& \vrule#& \hfil#\hfil& \vrule#& \hfil#\hfil& 
    \tabskip=0pt\vrule#\cr
  \noalign{\hrule}
    &\multispan5\hfil Just a table\hfil&\cr
  \noalign{\hrule}
    &one&&two&&three&\cr &a&&b&&c&\cr
  \noalign{\hrule}
    }
\end{verbatim}
The result of this is
\begin{disp}\PopListLevel
\leavevmode\message{single indent and sufficient vertical}%
\hbox{\leftskip0pt \rightskip0pt
 \vbox{\offinterlineskip
\tabskip=0pt \halign to \hsize{\strut
    \vrule#\tabskip=0pt plus 1fil\strut&
    \hfil#\hfil& \vrule#& \hfil#\hfil& 
                 \vrule#& \hfil#\hfil&
    \tabskip=0pt\vrule#\cr
    \noalign{\hrule}
    &\multispan5\hfil Just a table\hfil&\cr
    \noalign{\hrule}
    &one&&two&&three&\cr
    &a&&b&&c&\cr
    \noalign{\hrule}
    }}}\end{disp}
All of the vertical rules
of the table are in a separate column. This is the only way
to get the space around the items to stretch.

%%\point The alignment
%\section{The alignment}
%\point The alignment
\section{The alignment}

%After the template line any number of lines terminated by \cs{cr}
%can follow. \TeX\ reads all of these lines, processing the
%entries in order to find the maximal width (height) in 
%each column (row).
%Because all entries are kept in memory,
%long tables can overflow \TeX's main memory.
%For such tables it is better to write a special-purpose macro.
After the template line any number of lines terminated by \cs{cr}
can follow. \TeX\ reads all of these lines, processing the
entries in order to find the maximal width (height) in 
each column (row).
Because all entries are kept in memory,
long tables can overflow \TeX's main memory.
For such tables it is better to write a special-purpose macro.

%%\spoint Reading an entry
%\subsection{Reading an entry}
%\spoint Reading an entry
\subsection{Reading an entry}

%Entries in an alignment are composed of the 
%constant $u$ and $v$ parts
%of the template, and the variable $\alpha$ part.
%Basically \TeX\ forms the sequence of tokens $u\alpha v$
%and processes this. However, there are two special cases
%where \TeX\ has to expand before it forms this sequence.
Entries in an alignment are composed of the 
constant $u$ and $v$ parts
of the template, and the variable $\alpha$ part.
Basically \TeX\ forms the sequence of tokens $u\alpha v$
and processes this. However, there are two special cases
where \TeX\ has to expand before it forms this sequence.

%Above, the \cs{noalign} command was described. 
%Since this requires a different treatment from other
%alignment entries, 
%\TeX\ expands, after it has read a \cs{cr},
%the first token of the first $\alpha$ string
%of the next line to
%see whether that is or expands to \cs{noalign}.
%Similarly, for all entries
%in a line the first token is expanded to see
%whether it is or expands to \cs{omit}. This control sequence
%will be described below.
Above, the \cs{noalign} command was described. 
Since this requires a different treatment from other
alignment entries, 
\TeX\ expands, after it has read a \cs{cr},
the first token of the first $\alpha$ string
of the next line to
see whether that is or expands to \cs{noalign}.
Similarly, for all entries
in a line the first token is expanded to see
whether it is or expands to \cs{omit}. This control sequence
will be described below.

%Entries starting with an \cs{if...} conditional, or a macro
%expanding to one, may be misinterpreted owing to this
%premature expansion. For example,
%\begin{verbatim}
%\halign{$#$\cr \ifmmode a\else b\fi\cr}
%\end{verbatim}
%will give
%\begin{disp}\leavevmode
%     \vbox{\halign{$#$\cr \ifmmode a\else b\fi\cr}}\end{disp}
%because the conditional is evaluated before math mode has been set up.
%The solution is, as in many other cases, to insert a 
%\cs{relax} control sequence to stop the expansion.
%Here the \cs{relax} has to be inserted at the start of the 
%alignment entry.
Entries starting with an \cs{if...} conditional, or a macro
expanding to one, may be misinterpreted owing to this
premature expansion. For example,
\begin{verbatim}
\halign{$#$\cr \ifmmode a\else b\fi\cr}
\end{verbatim}
will give
\begin{disp}\leavevmode
     \vbox{\halign{$#$\cr \ifmmode a\else b\fi\cr}}\end{disp}
because the conditional is evaluated before math mode has been set up.
The solution is, as in many other cases, to insert a 
\cs{relax} control sequence to stop the expansion.
Here the \cs{relax} has to be inserted at the start of the 
alignment entry.

%If neither \cs{noalign} nor \cs{omit} (see below) is found,
%\TeX\ will process an input stream composed
%of  the $u$ part, the $\alpha$ tokens
%(which are delimited by either \n\& or \cs{span}, see below),
%and the $v$ part.
If neither \cs{noalign} nor \cs{omit} (see below) is found,
\TeX\ will process an input stream composed
of  the $u$ part, the $\alpha$ tokens
(which are delimited by either \n\& or \cs{span}, see below),
and the $v$ part.

%Entries are delimited by \n\&, \cs{span}, or \cs{cr}, but
%only if such a token occurs on the same level of grouping.
%This makes it possible to have an alignment as an entry of
%another alignment.
Entries are delimited by \n\&, \cs{span}, or \cs{cr}, but
only if such a token occurs on the same level of grouping.
This makes it possible to have an alignment as an entry of
another alignment.

%%\spoint Alternate specifications: \cs{omit}
%\subsection{Alternate specifications: \cs{omit}}
%\spoint Alternate specifications: \cs{omit}
\subsection{Alternate specifications: \cs{omit}}

%The template line will rarely be sufficient to describe
%all lines of the alignment. For lines where items should be
%set differently the command \csidx{omit} exists:
%if the first token in an entry is (or expands to) \cs{omit}
%the trivial template \n\# is used instead of
%what the template line specifies.
The template line will rarely be sufficient to describe
all lines of the alignment. For lines where items should be
set differently the command \csidx{omit} exists:
if the first token in an entry is (or expands to) \cs{omit}
the trivial template \n\# is used instead of
what the template line specifies.

%\begin{example} The following alignment uses the same template for
%all columns, but in the second column an \cs{omit} command
%is given.
%\begin{verbatim}
%\tabskip=1em
%\halign{&$<#>$\cr a&\omit (b)&c \cr}
%\end{verbatim}
%The output of this is
%\begin{disp}\leavevmode\vbox{\tabskip=1em
%\halign{&$<#>$\cr a&\omit (b)&c \cr}}
%\end{disp}
%\end{example}
\begin{example} The following alignment uses the same template for
all columns, but in the second column an \cs{omit} command
is given.
\begin{verbatim}
\tabskip=1em
\halign{&$<#>$\cr a&\omit (b)&c \cr}
\end{verbatim}
The output of this is
\begin{disp}\leavevmode\vbox{\tabskip=1em
\halign{&$<#>$\cr a&\omit (b)&c \cr}}
\end{disp}
\end{example}

%%\spoint Spanning across multiple columns: \cs{span}
%\subsection{Spanning across multiple columns: \cs{span}}
%\spoint Spanning across multiple columns: \cs{span}
\subsection{Spanning across multiple columns: \cs{span}}

%Sometimes it is desirable to have material spanning several
%columns. The most obvious example is that of a heading above
%a table. For this \TeX\ provides the \cs{span} command.
Sometimes it is desirable to have material spanning several
columns. The most obvious example is that of a heading above
a table. For this \TeX\ provides the \cs{span} command.

%Entries are delimited either by \n\&, by \cs{cr}, or by \csidx{span}.
%In the last case \TeX\ will omit the tabskip glue that
%would normally follow the entry thus delimited, and
%it will typeset the material just read plus the following
%entry in the joint space available.
Entries are delimited either by \n\&, by \cs{cr}, or by \csidx{span}.
In the last case \TeX\ will omit the tabskip glue that
would normally follow the entry thus delimited, and
it will typeset the material just read plus the following
entry in the joint space available.

%As an example,
%\begin{verbatim}
%\tabskip=1em
%\halign{&#\cr a&b&c&d\cr a&\hrulefill\span\hrulefill&d\cr}
%\end{verbatim}
%gives
%\begin{disp}\leavevmode\vbox{\tabskip=1em
%\halign{&#\cr a&b&c&d\cr a&\hrulefill\span\hrulefill&d\cr}}
%\end{disp} Note that there is no tabskip glue in between the
%two spanned columns, but there is tabskip glue before the
%\alt
%first column and after the last.
As an example,
\begin{verbatim}
\tabskip=1em
\halign{&#\cr a&b&c&d\cr a&\hrulefill\span\hrulefill&d\cr}
\end{verbatim}
gives
\begin{disp}\leavevmode\vbox{\tabskip=1em
\halign{&#\cr a&b&c&d\cr a&\hrulefill\span\hrulefill&d\cr}}
\end{disp} Note that there is no tabskip glue in between the
two spanned columns, but there is tabskip glue before the
\alt
first column and after the last.

%Using the \cs{omit} command this same alignment could
%have been generated as
%\begin{verbatim}
%\halign{&#\cr a&b&c&d\cr a&\hrulefill\span\omit&d\cr}
%\end{verbatim}
Using the \cs{omit} command this same alignment could
have been generated as
\begin{verbatim}
\halign{&#\cr a&b&c&d\cr a&\hrulefill\span\omit&d\cr}
\end{verbatim}

%The \cs{span}\cs{omit} combination is used in the
%plain \TeX\ macro
%\cs{multispan}: for instance
%\begin{disp}\cs{multispan4}\quad gives\quad \verb>\omit\span\omit\span\omit\span\omit>
%\end{disp} which spans across three tabs, and removes the templates
%of four entries. 
%Repeating the above example once again:
%\begin{verbatim}
%\halign{&#\cr a&b&c&d\cr a&\multispan2\hrulefill&d\cr}
%\end{verbatim}
%The argument of \cs{multispan} is a single token,
%not a number,
%so in order to span more than 9 columns the argument
%should be enclosed in braces, for instance \verb>\multispan{12}>.
%\alt
%Furthermore, a space after a single-digit argument
%will wind up in the output.
The \cs{span}\cs{omit} combination is used in the
plain \TeX\ macro
\cs{multispan}: for instance
\begin{disp}\cs{multispan4}\quad gives\quad \verb>\omit\span\omit\span\omit\span\omit>
\end{disp} which spans across three tabs, and removes the templates
of four entries. 
Repeating the above example once again:
\begin{verbatim}
\halign{&#\cr a&b&c&d\cr a&\multispan2\hrulefill&d\cr}
\end{verbatim}
The argument of \cs{multispan} is a single token,
not a number,
so in order to span more than 9 columns the argument
should be enclosed in braces, for instance \verb>\multispan{12}>.
\alt
Furthermore, a space after a single-digit argument
will wind up in the output.

%For a `low budget' solution to spanning columns plain \TeX\ has the
%macro \csidx{hidewidth}, defined by 
%\begin{verbatim}
%\newskip\hideskip \hideskip=-1000pt plus 1fill 
%\def\hidewidth{\hskip\hideskip}
%\end{verbatim}
%Putting \cs{hidewidth} at the beginning or end of an alignment entry
%will make its width zero, with the material in the entry
%sticking out to the left or right respectively.
For a `low budget' solution to spanning columns plain \TeX\ has the
macro \csidx{hidewidth}, defined by 
\begin{verbatim}
\newskip\hideskip \hideskip=-1000pt plus 1fill 
\def\hidewidth{\hskip\hideskip}
\end{verbatim}
Putting \cs{hidewidth} at the beginning or end of an alignment entry
will make its width zero, with the material in the entry
sticking out to the left or right respectively.


%%\spoint Rules in alignments
%\subsection{Rules in alignments}
%\index{alignments!rules in|(}
%\spoint Rules in alignments
\subsection{Rules in alignments}
\index{alignments!rules in|(}

%Horizontal rules inside a horizontal alignment will mostly
%\howto Draw rules in an alignment\par
%be across the width of the alignment. The easiest way
%to attain this is to use
%\begin{verbatim}
%\noalign{\hrule}
%\end{verbatim}
%lines inside the alignment. If the alignment is contained
%in a vertical box, lines above and below the alignment
%can be specified with
%\begin{verbatim}
%\vbox{\hrule \halign{...} \hrule}
%\end{verbatim}
%The most general way to get horizontal lines in an alignment
%is to use
%\cstoidx multispan\par
%\begin{disp}\cs{multispan}$\,n$\cs{hrulefill}\end{disp}
%which can be used to underline arbitrary adjacent columns.
Horizontal rules inside a horizontal alignment will mostly
\howto Draw rules in an alignment\par
be across the width of the alignment. The easiest way
to attain this is to use
\begin{verbatim}
\noalign{\hrule}
\end{verbatim}
lines inside the alignment. If the alignment is contained
in a vertical box, lines above and below the alignment
can be specified with
\begin{verbatim}
\vbox{\hrule \halign{...} \hrule}
\end{verbatim}
The most general way to get horizontal lines in an alignment
is to use
\cstoidx multispan\par
\begin{disp}\cs{multispan}$\,n$\cs{hrulefill}\end{disp}
which can be used to underline arbitrary adjacent columns.

%Vertical rules in alignments take some more care.
%Since a horizontal alignment breaks up into
%horizontal boxes that will be placed on a vertical list,
%\TeX\ will insert baselineskip glue in between the rows
%of the alignment. If vertical rules in adjacent rows
%are to abut, it is necessary to prevent baselineskip glue,
%for instance by the \cs{offinterlineskip} macro.
Vertical rules in alignments take some more care.
Since a horizontal alignment breaks up into
horizontal boxes that will be placed on a vertical list,
\TeX\ will insert baselineskip glue in between the rows
of the alignment. If vertical rules in adjacent rows
are to abut, it is necessary to prevent baselineskip glue,
for instance by the \cs{offinterlineskip} macro.

%In order to ensure that rows will still be properly spaced
%it is then necessary to place a {\italic strut\/}
%somewhere in the preamble.
%A~strut is an invisible object with a certain height
%and depth. Putting that in the preamble guarantees that
%every line will have at least that height and depth.
%In the plain format \csidx{strut} is
%defined statically as
%\begin{verbatim}
%\vrule height8.5pt depth3.5pt width0pt
%\end{verbatim}
%so this must be changed when other fonts or sizes are used.
In order to ensure that rows will still be properly spaced
it is then necessary to place a {\italic strut\/}
somewhere in the preamble.
A~strut is an invisible object with a certain height
and depth. Putting that in the preamble guarantees that
every line will have at least that height and depth.
In the plain format \csidx{strut} is
defined statically as
\begin{verbatim}
\vrule height8.5pt depth3.5pt width0pt
\end{verbatim}
so this must be changed when other fonts or sizes are used.

%It is a good idea to use a whole column for a~vertical
%rule, that is, to write
%\begin{verbatim}
%\vrule#&
%\end{verbatim}
%in the preamble and
%to leave the corresponding entry in the alignment empty.
%Omitting the vertical rule can then be done by specifying \cs{omit},
%and the size of the rule can be specified explicitly by
%putting, for instance,
%\hbox{\n{height 15pt}} in the entry instead of leaving
%it empty. Of course, tabskip glue will now be specified to the
%left and right of the rule, so some extra tabskip assignments
%may be needed in the preamble.
It is a good idea to use a whole column for a~vertical
rule, that is, to write
\begin{verbatim}
\vrule#&
\end{verbatim}
in the preamble and
to leave the corresponding entry in the alignment empty.
Omitting the vertical rule can then be done by specifying \cs{omit},
and the size of the rule can be specified explicitly by
putting, for instance,
\hbox{\n{height 15pt}} in the entry instead of leaving
it empty. Of course, tabskip glue will now be specified to the
left and right of the rule, so some extra tabskip assignments
may be needed in the preamble.

%\index{alignments!rules in|)}
\index{alignments!rules in|)}

%\subsection{End of a line: \cs{cr} and \cs{crcr}}
\subsection{End of a line: \cs{cr} and \cs{crcr}}

%All lines in an alignment are terminated by the \csidx{cr} control 
%sequence, including the last line. 
%\TeX\ is not  able to infer from
%a closing brace in the $\alpha$~part that the
%alignment has ended, because an unmatched
%closing brace is perfectly valid in
%an alignment entry;  it may match an opening brace in
%the $u$~part of the corresponding preamble entry.
All lines in an alignment are terminated by the \csidx{cr} control 
sequence, including the last line. 
\TeX\ is not  able to infer from
a closing brace in the $\alpha$~part that the
alignment has ended, because an unmatched
closing brace is perfectly valid in
an alignment entry;  it may match an opening brace in
the $u$~part of the corresponding preamble entry.

%\TeX\ has a primitive command \csidx{crcr} that is equivalent
%to \cs{cr}, but it has no effect if it immediately follows
%a~\cs{cr}.
%Consider as an example the definition in plain \TeX\
%of \csidx{cases}:
%\begin{verbatim}
%\def\cases#1{%
%    \left\{\,\vcenter{\normalbaselines\m@th
%              \ialign{ $##\hfil$& \quad##\hfil \crcr #1\crcr}}%
%    \right.}
%\end{verbatim}
%Because of the \cs{crcr} after the user argument \verb.#1.,
%the following two applications of this macro
%\begin{disp}\verb>\cases{1&2\cr 3&4}>\quad and\quad \verb>\cases{1&2\cr 3&4\cr}>\end{disp}
%both work. In the first case the \cs{crcr} in the macro
%definition ends the last line;
%in the second case the user's \cs{cr} ends the line,
%and the \cs{crcr} is redundant.
\TeX\ has a primitive command \csidx{crcr} that is equivalent
to \cs{cr}, but it has no effect if it immediately follows
a~\cs{cr}.
Consider as an example the definition in plain \TeX\
of \csidx{cases}:
\begin{verbatim}
\def\cases#1{%
    \left\{\,\vcenter{\normalbaselines\m@th
              \ialign{ $##\hfil$& \quad##\hfil \crcr #1\crcr}}%
    \right.}
\end{verbatim}
Because of the \cs{crcr} after the user argument \verb.#1.,
the following two applications of this macro
\begin{disp}\verb>\cases{1&2\cr 3&4}>\quad and\quad \verb>\cases{1&2\cr 3&4\cr}>\end{disp}
both work. In the first case the \cs{crcr} in the macro
definition ends the last line;
in the second case the user's \cs{cr} ends the line,
and the \cs{crcr} is redundant.

%After \cs{cr} and after a non-redundant \cs{crcr} the
%\gr{token parameter} \csidx{everycr} is inserted.
%This includes the \cs{cr} terminating the template line.
After \cs{cr} and after a non-redundant \cs{crcr} the
\gr{token parameter} \csidx{everycr} is inserted.
This includes the \cs{cr} terminating the template line.

%%\point Example: math alignments
%\section{Example: math alignments}
%\point Example: math alignments
\section{Example: math alignments}

%The plain format has several alignment macros that function
%in math mode. One example is \csidx{matrix}, defined by
%\begin{verbatim}
%\def\matrix#1{\null\,\vcenter{\normalbaselines\m@th
%    \ialign{\hfil$##$\hfil && \quad\hfil$##$\hfil\crcr
%            \mathstrut\crcr
%         \noalign{\kern-\baselineskip}
%            #1\crcr
%            \mathstrut\crcr
%         \noalign{\kern-\baselineskip}}}\,}
%\end{verbatim}
%This uses a repeating (starting with~\verb>&&>) second preamble entry;
%each entry is centred by an \cs{hfil} before and after it,
%and there is a \cs{quad} of space in between columns.
%Tabskip glue was not used for this, because there should not
%be any glue preceding or following the matrix.
The plain format has several alignment macros that function
in math mode. One example is \csidx{matrix}, defined by
\begin{verbatim}
\def\matrix#1{\null\,\vcenter{\normalbaselines\m@th
    \ialign{\hfil$##$\hfil && \quad\hfil$##$\hfil\crcr
            \mathstrut\crcr
         \noalign{\kern-\baselineskip}
            #1\crcr
            \mathstrut\crcr
         \noalign{\kern-\baselineskip}}}\,}
\end{verbatim}
This uses a repeating (starting with~\verb>&&>) second preamble entry;
each entry is centred by an \cs{hfil} before and after it,
and there is a \cs{quad} of space in between columns.
Tabskip glue was not used for this, because there should not
be any glue preceding or following the matrix.

%The combination of a \cs{mathstrut} and \verb>\kern-\baselineskip>
%above and below the matrix increases the vertical size
%such that two matrices with the same number of rows will have
%the same height and depth, which would not otherwise be the case
%if one of them had subscripts in the last row, but the other
%not. The \cs{mathstrut} causes interline glue to be inserted 
%and, because it has a size equal to \cs{baselineskip},
%the negative kern will effectively leave only the interline glue,
%thereby buffering any differences in the first and last line.
%Only to a certain point, of course: objects bigger than the
%opening brace will still result in a different height or depth of the 
%matrix.
The combination of a \cs{mathstrut} and \verb>\kern-\baselineskip>
above and below the matrix increases the vertical size
such that two matrices with the same number of rows will have
the same height and depth, which would not otherwise be the case
if one of them had subscripts in the last row, but the other
not. The \cs{mathstrut} causes interline glue to be inserted 
and, because it has a size equal to \cs{baselineskip},
the negative kern will effectively leave only the interline glue,
thereby buffering any differences in the first and last line.
Only to a certain point, of course: objects bigger than the
opening brace will still result in a different height or depth of the 
matrix.

%Another, more
%complicated, example of an alignment for math mode is \cs{eq\-alignno}.
%\cstoidx eqalignno\par\cstoidx centering\par
%\begin{verbatim}
%\def\eqalignno#1{\begin{disp}l@y \tabskip\centering
%  \halign to\displaywidth{
%    \hfil$\@lign\displaystyle{##}$%    -- first column
%        \tabskip\z@skip
%    &$\@lign\displaystyle{{}##}$\hfil% -- second column
%        \tabskip\centering
%    &\llap{$\@lign##$}%                -- third column
%        \tabskip\z@skip\crcr   %   end of the preamble
%       #1\crcr}}
%\end{verbatim}
%Firstly, the tabskip is set to zero after the equation
%number, so this number is set flush with the right margin.
%Since it is placed by \cs{llap}, its effective width
%is zero. Secondly, the tabskip between the
%first and second columns is also zero, and the tabskip
%before the first column and after the second  is
%\alt
%\cs{centering}, which is \n{0pt plus 1000pt minus 1000pt},
%so the first column and second  are jointly centred
%in the \cs{hsize}. Note that, because of the
%\n{minus 1000pt}, these two columns will happily go
%outside the left and right margins, overwriting any
%equation numbers.
Another, more
complicated, example of an alignment for math mode is \cs{eq\-alignno}.
\cstoidx eqalignno\par\cstoidx centering\par
\begin{verbatim}
\def\eqalignno#1{\begin{disp}l@y \tabskip\centering
  \halign to\displaywidth{
    \hfil$\@lign\displaystyle{##}$%    -- first column
        \tabskip\z@skip
    &$\@lign\displaystyle{{}##}$\hfil% -- second column
        \tabskip\centering
    &\llap{$\@lign##$}%                -- third column
        \tabskip\z@skip\crcr   %   end of the preamble
       #1\crcr}}
\end{verbatim}
Firstly, the tabskip is set to zero after the equation
number, so this number is set flush with the right margin.
Since it is placed by \cs{llap}, its effective width
is zero. Secondly, the tabskip between the
first and second columns is also zero, and the tabskip
before the first column and after the second  is
\alt
\cs{centering}, which is \n{0pt plus 1000pt minus 1000pt},
so the first column and second  are jointly centred
in the \cs{hsize}. Note that, because of the
\n{minus 1000pt}, these two columns will happily go
outside the left and right margins, overwriting any
equation numbers.

%\index{alignments|)}
%\endofchapter
\index{alignments|)}
\endofchapter

\end{document}
